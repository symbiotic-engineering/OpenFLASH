%\documentclass[lineno]{jfm}
\documentclass[lineno]{JFM-FLM_Au}

%\usepackage{showframe}
\usepackage{pdflscape}
\usepackage{array}
\usepackage{amsmath,amssymb}
\usepackage{multirow}


\newtheorem{lemma}{Lemma}
\newtheorem{corollary}{Corollary}

\lefttitle{R. McCabe, K. Khanal, Y. Bimali, E. Lo, C. Treacy, and M. N. Haji}
\righttitle{Journal of Fluid Mechanics}

\title{Numerics of the matched eigen-function method for computing wave forces on concentric bodies}

% Numerics of using matched eigen-function expansions to compute wave forces on concentric bodies

\author{Rebecca McCabe\aff{1}, 
Kapil Khanal\aff{2},
Yinghui Bimali\aff{3},
En Lo\aff{4,5},
Collin Treacy\aff{6},
\and 
Maha Haji\aff{6}}

\affiliation{
\aff{1}Sibley School of Mechanical and Aerospace Engineering, Cornell University, 124 Hoy Road, Ithaca, NY 14853, USA
\aff{2}Department of Systems Engineering, Cornell University, 136 Hoy Road, Ithaca, NY 14853, USA
\aff{3} AEP cornell (todo)
\aff{4} enviro eng cornell (todo)
\aff{5} oxford (todo)
\aff{6} meche Michigan (todo)
}

\corresau{Rebecca McCabe, \email{rgm222@cornell.edu}}

\begin{document}
\maketitle

\begin{abstract}
Offshore structures are important amidst the growing marine economy for sustainable energy and food.
To ensure survival of these structures, wave loads are quantified using hydrodynamic coefficients.
Calculating these added mass, damping, and excitation coefficients with the traditional boundary element method presents a computational bottleneck, limiting the application of advanced design procedures like optimization.
The matched eigen-function expansion method (MEEM), a long-known but rarely-used alternative, offers computational benefits due to its semi-analytical nature.
The authors use modern computing tools to implement an existing matched eigen-function solution for the radiation of surface-piercing compound annular cylinders with continuous and radially-monotonic body profiles, releasing it in an open-source toolbox called OpenFLASH.
This paper systematically investigates numerical properties of the OpenFLASH MEEM implementation including accuracy, convergence, conditioning, and runtime.
Furthermore, it documents mathematical and computational subtleties not previously detailed, including alternate expressions for edge case geometries that otherwise yield indeterminate forms, and techniques that leverage convergence insights to improve model speed while preserving accuracy. 
Comparisons to the Capytaine boundary element method software show that OpenFLASH has XX\% accuracy, XX times faster runtime, XX better convergence, and XX times less memory use. 
Performance depends on geometry: larger diameters converge more slowly, and larger drafts require scaling to prevent numerical overflow.
Finally, the paper introduces a novel correction factor to extend the method to bodies with slanted (conical) surfaces, reducing the error for a frustum body from XX\% to XX\%. 
This correction factor, if combined with known extensions for moonpools and multi-valued vertical profiles, broadens the method's applicability to all concentric bodies of arbitrary profile, including concentric multi-body systems.
These contributions enable hydrodynamic analysis of a broad range of shapes with increased speed and confidence in the numerical accuracy, paving the way for future studies to apply optimization and yield improved designs.


Full author instructions can be found on the \href{https://www.cambridge.org/core/journals/journal-of-fluid-mechanics/information/author-instructions}{JFM website}. All papers should feature a single-paragraph abstract of no more than 250 words which must not spill onto the second page of the manuscript.
\end{abstract}

\begin{keywords}
Authors should not enter keywords on the manuscript, as these must be chosen by the author during the online submission process and will then be added during the typesetting process (see \href{https://www.cambridge.org/core/journals/journal-of-fluid-mechanics/information/list-of-keywords}{Keyword PDF} for the full list).  Other classifications will be added at the same time.
\end{keywords}

%{\bf MSC Codes }  {\it(Optional)} Please enter your MSC Codes here

\section{Introduction (\textcolor{teal}{Becca})}
\label{sec:Introduction}
% this is the UMERC abstract copy-pasted
% Floating body hydrodynamics are typically solved numerically using the boundary element method.
% The associated code is computationally costly, scaling with the number of mesh panels, and can have accuracy issues at specific frequencies and for thin bodies.
% In this work, we instead implement a previously-developed matched eigenfunction expansion method to semi-analytically solve the linear potential flow radiation problem for axisymmetric bodies.
% This method first establishes distinct fluid regions based on the body geometry and expresses the velocity potential as a function of vertical and radial basis functions (eigenfunctions) with unknown coefficients.
% Eigenfunctions are chosen to automatically enforce several boundary conditions of the problem.
% The coefficients are found by truncating and solving an infinite linear system representing the matching of potential and radial velocity across fluid region boundaries.
% This yields a solution for the 3D potential and the hydrodynamic coefficients.
% We compare the results and computational complexity of the matched eigenfunction expansion method with that of the standard boundary element method.
% Benefits of the former include 10x faster solve time and lack of meshing, which are particularly appealing in optimization workflows. 
% Our framework is released as an open-source python package to enable future integration with design tools, implementation of gradients, and democratization of this efficient method.
% This is a meaningful contribution because prior relevant implementations of the matched eigenfunction expansion method are, to the authors’ knowledge, private and not available open-source or even commercially.
% Future work will extend this formulation to different kinds of bodies and arrays.
% end UMERC abstract
Linear potential flow theory is a simplification of the Navier-Stokes equations widely used to model wave-structure interactions.
Early analytical solutions for the wave force on vertical cylindrical bodies under linear potential flow theory include Havelock's 1940? solution for infinite cylinders in infinite depth (cite), and McCamy and Fuchs' 1956? solution for bottom-mounted cylinders in finite depth (cite).
Both take advantage of the separability of the Laplace equation in cylindrical coordinates.
Fully analytical solutions are not available for truncated cylinders due to the impossibility of analytically enforcing continuity of potential between the water under the body and the water surrounding it.

In 1980, Yeung developed a semi-analytical method called the matched eigen-function expansion method (MEEM) to address this challenge (cite).
The solution in each fluid region is obtained analytically in terms of unknown coefficients, and these coefficients are found by enforcing continuity of potential and radial velocity at the region boundaries.
The method has been extended to cover axisymmetric geometries with multiple concentric cylinders (\cite{kokkinowrachos_behaviour_1986}), damping plates (\cite{olaya_hydrodynamic_2015}), two concentric cylinders (\cite{mavrakos_hydrodynamic_2004} and \cite{chau2012inertia}), three concentric cylinders (\cite{zhang_performance_2024}), and has recently been thoroughly covered in an analytical hydrodynamics textbook (\cite{chatzigeorgiou2018analytical}).

Despite this interest from hydrodynamics researchers, MEEM has not been widely adopted by the offshore engineering community, likely due to the complexity of its implementation and the rise of boundary element method (BEM) software packages that are easier to use and applicable to arbitrary geometries, albeit more computationally costly.
With the growing need for efficient hydrodynamic analysis tools in design optimization of offshore renewable energy, interest in MEEM has resurged, although the lack of an accessible computational implementation and the numerical properties thereof has been a barrier.

We have implemented MEEM for concentric surface-piercing cylinders in an open-source python package called OpenFLASH (Flexible Library for Analytical and Semi-analytical Hydrodynamics).
This paper serves to document the mathematical formulation, matrix representation, and numerical properties of the method, augmenting the authors' preliminary work \cite{mccabe_open-source_2024}.

Section~\ref{sec:mathematical-formulation} covers the formulation, mainly unifying the aforementioned literature in a consistent notation, explaining the method for an audience unfamiliar with analytical hydrodynamics, clarifying the block matrix structure and asymptotic behavior for small and large frequencies, and discussing numerical subtleties needed to avoid overflow and finite precision effects.
Section~\ref{sec:convergence} details the novel results of a convergence study and provides guidance on the number of terms required to achieve a desired accuracy for a given geometric configuration.
Section~\ref{sec:slant} introduces a novel adjustment to the formulation to approximate conically slanted geometries.
Section~\ref{sec:validation} compares the results of OpenFLASH to those of Capytaine, a widely-used BEM package, and section~\ref{sec:compute-time} benchmarks computational runtime and accuracy.
We conclude in section~\ref{sec:conclusion} with a summary of findings and outlook on future work.

\section{Mathematical Formulation}\label{sec:mathematical-formulation}
\subsection{Linear Hydrodynamics and eigen-functions \textcolor{blue}{Collin}}

The floating body dynamics problem of interest is finding the hydrodynamic forces on a series of $M$ heaving surface-piercing compound annular cylinders, as shown in Fig.~\ref{fig:Diagram}. The internal fluid regions underneath each cylindrical ring are denoted by $i_1$, $i_2$, ..., $i_M$. The external fluid region surrounding the body is denoted by $e$. The geometry of the $m$th internal region is defined in terms of the radius $a_m$ measured from the axis of symmetry and the draft $d_m$ measured from the mean free surface. The origin $O$ is located at the intersection of the axis of symmetry of the body and the mean free surface of the fluid. It is assumed that $a_{m+1}>a_m$ and $d_m<h$ for all $m$, where $h$ is the sea depth. 

\begin{figure}
    \centering
    \includegraphics[width=0.95\linewidth]{figs/Multi MEEM Diagram.pdf}
    \caption{Side view of concentric cylindrical bodies.}
    \label{fig:Diagram}
\end{figure} 

To model the fluid in the internal and external regions in this work, linear potential flow theory is used. The fluid is assumed to be inviscid, irrotational, and incompressible. Additionally, the wave amplitude is assumed to be small relative to the wavelength, and all motion of the body is small. With these assumptions, the Navier-Stokes equations simplify to the Laplace equation $\nabla^2 \Phi_\mathrm{T}(\mathbf{x},t)=0$, where the fluid velocity is $\mathbf{v} = \nabla \Phi_\mathrm{T}(\mathbf{x},t)$. The potential function $\Phi_\mathrm{T}$ is both a function of spatial coordinates $\mathbf{x}$ and time $t$. Since only axisymmetric bodies are modeled in this work, the cylindrical coordinate system $\mathbf{x}=(r, \theta, z)$ is used, as shown in Fig.~\ref{fig:Diagram}. The spacial and temporal dependencies can be separated as $\Phi_\mathrm{T}(\mathbf{x},t) = \mathrm{Re} \{ \phi_\mathrm{T}(\mathbf{x})e^{-i \omega t}\}$, where $\phi_\mathrm{T} \in \mathbb{C}$ and $\omega$ is the angular frequency of the incident wave. 

The total complex potential $\phi_\mathrm{T}$ can be written as the superposition of the incident $\phi_\mathrm{I}$, diffracted $\phi_\mathrm{D}$, and radiated $\phi_\mathrm{R}$ potentials. The incident wave potential for a regular wave propagating in a direction at an angle $\beta$ from the x-axis is
\begin{equation}
    \phi_\mathrm{I} = -i \frac{g A}{\omega} \frac{\cosh k (z+h)}{\cosh k h}  e^{ikr\cos(\theta- \beta)}
\end{equation}
where $g$ is the acceleration due to gravity, $A$ is the wave amplitude, $\omega$ is the angular wave frequency, $k$ is the wave number, and $h$ is the water depth~\cite{chatzigeorgiou2018analytical}. The angel $\beta$ is set to zero without loss of generality, as the hydrodynamic forces are invariant with the incident wave direction for axisymmetric bodies restricted to heave motion exclusively. The wave number and angular frequency are related by the dispersion relation
\begin{equation}
    k \tanh(kh)=\frac{\omega^2}{g}.
\end{equation}

The diffracted and radiated wave potentials can be solved by finding velocity potentials that satisfy the boundary conditions on the wetted surface of the body. However, to find the hydrodynamic forces on the floating body, only the incident and radiated potential are needed, as explained in Sec.~\ref{Hydrodynamic Forces}. Thus, the main focus of Sec.~\ref{Matching Across Fluid Boundaries} and~\ref{Block Matrix Structure} is solving the radiation problem. For simplicity, $\phi$ is used throughout the rest of this work to denote the complex radiated potential instead of $\phi_\mathrm{R}$. 

To solve the radiation problem, the radiated potential $\phi$ is defined separately for the internal and external regions as $\phi^{i_m}$ and $\phi^{e}$, respectively. The velocity potentials for the internal regions must satisfy
\begin{equation}\label{Laplace equation internal region}
    \nabla^2\phi^{i_m}=0,   
\end{equation}
\begin{equation}\label{No flux BC at wetted surface}
    \left. \frac{\partial\phi^{i_m}}{\partial z} \right|_{z=-d_m}=\left\{\begin{matrix}
    0 \text{ when region } m \text{ is fixed} \\
 1 \text{ when region } m \text{ is heaving}
\end{matrix}\right.,
\end{equation}
and
\begin{equation}\label{No flux BC at sea floor internal}
    \left. \frac{\partial\phi^{i_m}}{\partial z} \right|_{z=-h}=0,   
\end{equation}
while the velocity potential for the external region must satisfy
\begin{equation}\label{Laplace equation external region}
    \nabla^2\phi^{e}=0,   
\end{equation}
\begin{equation}\label{Free surface BC}
    \left( -\frac{\omega^2}{g}\phi^{e} + \left. \frac{\partial\phi^{e}}{\partial z} \right) \right|_{z=0}= 0,
\end{equation}
and
\begin{equation}\label{No flux BC at sea floor external}
    \left. \frac{\partial\phi^{e}}{\partial z} \right|_{z=-h}=0.   
\end{equation}
The velocity potential in the interior regions can be written as the superposition of a homogeneous part $\phi^{i_m}_\mathrm{h}$, which corresponds to the first case of Eq.~\ref{No flux BC at wetted surface} when the body in region $m$ is fixed, and a particular part $\phi^{i_m}_\mathrm{p}$, which corresponds to the second case of Eq.~\ref{No flux BC at wetted surface} when the body in region $m$ is heaving with unit amplitude velocity. Thus, the total velocity potential in the $m$th region is $\phi^{i_m}= \phi^{i_m}_\mathrm{h} + \phi^{i_m}_\mathrm{p}$, where both $\phi^{i_m}_\mathrm{h}$ and $\phi^{i_m}_\mathrm{h}$ individually satisfy Eq.~\ref{Laplace equation internal region} and~\ref{No flux BC at sea floor internal}, and different cases of Eq.~\ref{No flux BC at wetted surface}. 

The Laplace equations in Eq.~\ref{Laplace equation internal region} and~\ref{Laplace equation external region} can be solved using separation of variables in cylindrical coordinates. The velocity potentials can be written as a product of eigen-functions $\phi (r, \theta, z)= R(r)\Theta(\theta)Z(z)$, where $\phi$ is $\phi^{i_m}$ and $\phi^{e}$ for the internal and external regions, respectively. Substituting this into the Laplace equation and separating each variable yields the following system of ordinary differential equations:
\begin{equation}\label{Z ODE}
    \frac{\mathrm{d}^2 Z}{\mathrm{d}z^2}-\lambda^2Z=0
\end{equation}
\begin{equation}\label{Theta ODE}
    \frac{\mathrm{d}^2 \Theta}{\mathrm{d}\theta^2}+\nu^2\Theta=0
\end{equation}
\begin{equation}\label{R ODE}
    r^2\frac{\mathrm{d}^2 R}{\mathrm{d}r^2}+r\frac{\mathrm{d} R}{\mathrm{d}r} + (\lambda^2r^2-\nu^2)R=0.
\end{equation}

Since we are considering an axisymmetric geometry and only heave motion, $\Theta=1$ and $\nu=0$. This leaves the vertical and radial eigen-functions to be determined. There are two cases of the eigen-value $\lambda$ to consider: $\lambda \in \mathbb{R}$ and $\lambda \in \mathbb{I}$ ~\cite{chatzigeorgiou2018analytical}. In the first case, Eq.~\ref{R ODE} is the Bessel differential equation with Bessel and Hankel functions of the first and second kinds being solutions. In the second case, Eq.~\ref{R ODE} is the modified Bessel differential equation with modified Bessel functions of the first and second kinds being solutions. The functions for the velocity potential, eigen-functions, and eigen-values are summarized in Table~\ref{tab:MEEM-eigen-functions}. These solutions were chosen since they satisfy Eq.~\ref{Laplace equation internal region}-\ref{No flux BC at sea floor external} for their corresponding region. Note that, when modeling a single cylindrical ring heaving while all other regions are fixed, $\phi^{i_m}_\mathrm{p}$ will only be non-zero for the heaving region. This is indicated by the separate cases of the particular potentials shown in Table~\ref{tab:MEEM-eigen-functions}. $\textrm{I}_0$,  $\textrm{K}_0$, and $\textrm{H}_0^1$ are the zeroth-order modified Bessel function of the first kind, modified Bessel function of the second kind, and Henkel function of the first kind. The expression for $N_k$ is defined as:

\begin{equation}
    N_k = \frac{1}{2}\left(1+\frac{f_k}{2\lambda_k^eh}  \right)~
    \textrm{ where }
    f_k = 
    \begin{cases}
        \sinh(2 \lambda_0^e h), & k=0 \\ \sin(2\lambda_k^eh), & k\geq1
    \end{cases}.
\end{equation}
As shown in Table~\ref{tab:MEEM-eigen-functions}, when specifying the frequency $\omega$ and geometry of the cylindrical rings, the eigen-coefficients in the series representation of the homogeneous potentials are the only quantities left to be determined.

\begin{landscape}
\begin{table}
    \centering
    \begin{tabular}{|>{\centering\arraybackslash}p{0.085\linewidth}|>{\centering\arraybackslash}p{0.26\linewidth}|>{\centering\arraybackslash}p{0.34\linewidth}|>{\centering\arraybackslash}p{0.32\linewidth}|} \hline 
         Region&  $i_1$&  $i_m$ for $m>1$& $e$\\ \hline 
         Homog. potential &  $\phi^{i_1}_\mathrm{h}(r,z) = \displaystyle\sum_n^{N^{i_1}} C_{1n}^{i_1} R_{1n}^{i_1}(r) Z_n^{i_1}(z)$&  $\phi^{i_m}_\mathrm{h}(r,z) = \displaystyle\sum_{n}^{N^{i_m}} \left(C_{1n}^{i_m} R_{1n}^{i_m}(r) + C_{2n}^{i_m} R_{2n}^{i_m}(r) \right) Z_{n}^{i_m}(z)$& $\phi^{e}_\mathrm{h}(r,z) = \displaystyle\sum_n^{N^{e}} C_{1n}^{e} R_{1n}^{e}(r) Z_n^{e}(z)$\\ \hline 
         Partic. potential &  $\phi^{i_1}_\mathrm{p}(r,z) =  \begin{cases} \displaystyle\frac{1}{2(h-d_1)}\left[ (z+h)^2 - \frac{r^2}{2}\right] & \text{Heaving} \\ 0 & \text{Fixed}             
         \end{cases}$&  $\phi^{i_m}_\mathrm{p}(r,z) = \begin{cases} \displaystyle\frac{1}{2(h-d_m)}\left[ (z+h)^2 - \frac{r^2}{2}\right] & \text{Heaving} \\ 0 & \text{Fixed}       
         \end{cases}$& $0$\\ \hline 
         Radial eigen-function &  $R_{1n}^{i_1}(r) = \begin{cases}
            \frac{1}{2} &  n=0 \\[1em]   %%% <--- here
            \frac{\mathrm{I}_0(\lambda_{n}^{i_1}r)}{\mathrm{I}_0(\lambda_{n}^{i_1}a_1)} & n \ge 1
        \end{cases} $&  \shortstack{$R_{1n}^{i_m}(r) = \begin{cases}
            \frac{1}{2} &  n=0 \\[1em]   %%% <--- here
            \frac{\mathrm{I}_0(\lambda_{n}^{i_m}r)}{\mathrm{I}_0(\lambda_{n}^{i_m}a_m)} & n \ge 1
        \end{cases}$   \\ $R_{2n}^{i_m}(r) = \begin{cases}
           \frac{1}{2}\ln(\frac{r}{a_m}) &  n = 0 \\
        \frac{\mathrm{K}_0(\lambda_{n}^{i_m}r)}{\mathrm{K}_0(\lambda_{n}^{i_m}a_m)} & n \ge 1
        \end{cases}$}& $R_{1n}^{e}(r) = \begin{cases}
           \frac{\mathrm{H}_0^{1}(\lambda_0^e r)}{\mathrm{H}_0^{1}(\lambda_0^e a_M)} & n = 0 \\[1em] 
          \frac{\mathrm{K}_0(\lambda_n^e r)}{\mathrm{K}_0(\lambda_n^ea_M)} &  n \ge 1
        \end{cases}$\\ \hline 
 Vertical eigen-function & $Z_n^{i_1}(z) = \begin{cases}
           1 & n=0 \\[1em]   %%% <--- here
           \sqrt{2}\cos(\lambda_{n}^{i_1}(z+h)) & n \ge 1
        \end{cases}$& $Z_{n}^{i_m}(z) = \begin{cases}
           1 & n=0 \\[1em]   %%% <--- here
           \sqrt{2}\cos(\lambda_{n}^{i_m}(z+h)) & n \ge 1
        \end{cases}$&$    Z^{e}_{n}(z) = \begin{cases}
           N_0^{-\frac{1}{2}}\cosh( \lambda_0^{e}(z+h)) &  n=0 \\[1em]   %%% <--- here
           N_n^{-\frac{1}{2}}\cos( \lambda_n^{e}(z+h)) &  n \ge 1
        \end{cases}$\\ \hline
 Eigen-value& $\displaystyle \lambda_{n}^{i_1} = \frac{n\pi}{h-d_{1}},  n \geq 1$& $\displaystyle \lambda_{n}^{i_m} = \frac{n\pi}{h-d_m}, n \geq 1$&
 $\displaystyle \begin{cases} \lambda_{0}^{e} \tanh(\lambda_{0}^{e} h)= \omega^2/g, & n=0 \\ \lambda_{n}^{e} \tan(\lambda_{n}^{e} h) = -\omega^2/g, & n \geq 1\\ \end{cases} $\\\hline
    \end{tabular}
    \caption{Equations for the potential (homogeneous and particular), eigen-functions (radial and vertical), and eigen-values for each region. $i$ and $e$ denote internal and external regions.}
    \label{tab:MEEM-eigen-functions}

\end{table}
\end{landscape}

\subsection{Matching Across Fluid Boundaries \textcolor{blue}{Collin}}\label{Matching Across Fluid Boundaries}

The eigen-coefficients must be selected to enforce the radial velocity body boundary condition and the matching of the potentials and radial velocities at the edges of each region, earning this technique the name Matched Eigen-function Expansion Method (MEEM). As shown by the homogeneous and particular potential functions in Table.~\ref{tab:MEEM-eigen-functions}, the radiated potential is a sum of products of eigen-functions with unknown eigen-coefficients $C_{1n}^{i_m}$, $C_{2n}^{i_m}$, and $C_{1n}^{e}$. Note that $C_{2n}^{i_1}=C_{2n}^{e}=0$ for all $n$. In practice, the infinite series for the $i_1$, $i_m$ for $m>1$, and $e$ regions can be truncated at $N^{i_1}$, $N^{i_m}$, and $N^e$, yielding a total of $N_\mathrm{T} =N^{i_1}+\sum_{m=2}^{M}2N^{i_m}+N^e$ unknown eigen-coefficients. To solve for these unknowns, radial boundary conditions for each region must be imposed. 

For a geometry with $M$ internal regions (and one external region), there are $M$ vertical boundaries. Since both the value of the potential and fluid velocity must match at the boundary of neighboring regions, there are a total of $2M$ matching equations. However, these equations alone are not enough to solve for all eigen coefficients since the number of unknowns is greater than the number of equations ($N_\mathrm{T} >2M$). To generate an equal number of equations as unknowns, the orthogonality of the vertical eigen-functions can be leveraged to isolate the unknown coefficients in the finite series. 

Consider a generic function $Y(x)$ expressed as a series with coefficients $\alpha$ and basis functions $e(x)$: $Y(x)=\sum_i \alpha_i e_i(x)$.
If $e_j(x)$ is orthogonal to $e_i(x)$ from $x = a$ to $b$, then:

\begin{equation}
\begin{aligned}
       & \int\limits_a^b Y(x)e_j(x)dx = (b-a) <Y,e_j>=(b-a)<\sum_i \alpha_i e_i, e_j> \\ &= (b-a) \sum_i \alpha_i <e_i,e_j> = (b-a) \sum_i \alpha_i \delta_{ij} = (b-a) \alpha_j
\end{aligned}
\end{equation}

\noindent where $<\cdot,\cdot>$ is the inner product and $\delta_{ij}$ is Kronecker’s delta.
In the current hydrodynamics problem, the basis functions are the vertical eigen-functions $Z_n^{i_m}$ and $Z_n^{e}$.
Orthogonality of each eigen-function can be verified with the inner product.
In the first region, for example, $<Z_{1}^{i1},Z_{2}^{i1}>=\delta_{12}$.
Note that eigen-functions of different domains are not orthogonal, and their inner products will be expressed as coupling integrals.

\begin{figure}
    \centering
    \includegraphics[width=0.5\linewidth]{figs/Matching Diagram.pdf}
    \caption{Side view of taller and shorter region.}
    \label{fig:Matching Diagram}
\end{figure} 

The general formulation of the matching equations can be illustrated by considering two neighboring fluid regions, as shown in Fig.~\ref{fig:Matching Diagram}. The two fluid regions will be defined as the taller $\mathrm{t}$ and shorter $\mathrm{s}$ fluid regions. When considering the vertical boundary dividing the two regions at $r=a$, there are three different conditions to enforce: 1) the value of the velocity potentials at fluid-fluid boundaries are equal
\begin{equation}\label{potential matching}
    \phi^\mathrm{t}(a,z)=\phi^\mathrm{s}(a,z) \text{ for } -h \le z \le -d_\mathrm{s},
\end{equation}
2) the radial fluid velocities at the fluid-fluid boundary are equal
\begin{equation}\label{velocity matching}
    \frac{\partial \phi^\mathrm{s}}{\partial r}(a,z)=\frac{\partial \phi^\mathrm{t}}{\partial r}(a,z) \text{ for } -h \le z \le -d_\mathrm{s},
\end{equation}
and 3) the radial fluid velocity at the body-fluid boundary is zero
\begin{equation}\label{no radial velocity}
    \frac{\partial \phi^\mathrm{t}}{\partial r}(a,z)=0 \text{ for } -d_\mathrm{s} \le z \le -d_\mathrm{t}.
\end{equation}
\noindent To derive a system of linear algebraic equations in terms of the eigen-coefficients, we first multiply both sides of Eq.~\ref{potential matching} and~\ref{velocity matching} by a vertical eigen-function, $Z^\mathrm{s}_{k}(z)$ for potential matching and $Z^\mathrm{t}_{k}(z)$ for velocity matching, and integrate over the fluid-fluid boundary to get
\begin{equation}\label{potential matching integral}
    \int_{-h}^{-d_\mathrm{s}}\phi^\mathrm{t}(a,z) Z^\mathrm{s}_{k}(z) \mathrm{d}z=\int_{-h}^{-d_\mathrm{s}}\phi^\mathrm{s}(a,z) Z^\mathrm{s}_{k}(z) \mathrm{d}z
\end{equation}
and
\begin{equation}\label{velocity matching integral}
    \int_{-h}^{-d_\mathrm{s}}\frac{\partial \phi^\mathrm{t}}{\partial r}(a,z) Z^\mathrm{t}_{k}(z) \mathrm{d}z=\int_{-h}^{-d_\mathrm{s}}\frac{\partial \phi^\mathrm{s}}{\partial r}(a,z) Z^\mathrm{t}_{k}(z) \mathrm{d}z.
\end{equation}
Notice that the integrand of Eq.~\ref{velocity matching integral} is zero in the interval $-d_\mathrm{s} \le z \le -d_\mathrm{t}$, due to Eq.~\ref{no radial velocity}. For this reason, we can change the integration bounds on the left-hand-side of Eq.~\ref{velocity matching integral} from $-h \le z \le -d_\mathrm{s}$ to $-h \le z \le -d_\mathrm{t}$
\begin{equation}\label{modified velocity matching integral}
    \int_{-h}^{-d_\mathrm{t}}\frac{\partial \phi^\mathrm{t}}{\partial r}(a,z)Z^\mathrm{t}_{k}(z) \mathrm{d}z=\int_{-h}^{-d_\mathrm{s}}\frac{\partial \phi^\mathrm{s}}{\partial r}(a,z) Z^\mathrm{t}_{k}(z) \mathrm{d}z.
\end{equation}
Thus, Eq.~\ref{no radial velocity} and~\ref{velocity matching integral} are simultaneously enforced by Eq.~\ref{modified velocity matching integral}, while Eq.~\ref{potential matching} is enforced by Eq.~\ref{potential matching integral}. Next, the velocity potentials of each region can be rewritten in terms of their homogeneous and particular solutions as $\phi^\mathrm{t}=\phi^\mathrm{t}_\mathrm{h} + \phi^\mathrm{t}_\mathrm{p}$ and $\phi^\mathrm{s}=\phi^\mathrm{s}_\mathrm{h} + \phi^\mathrm{s}_\mathrm{p}$. Substituting these into Eq.~\ref{potential matching integral} and~\ref{modified velocity matching integral}, and moving all homogeneous and particular potentials to the left and right-hand-side, respectively, yields
\begin{multline}\label{separated potential matching integral}
    \int_{-h}^{-d_\mathrm{s}} \phi^\mathrm{s}_\mathrm{h}(a,z) Z^\mathrm{s}_{k}(z) \mathrm{d}z-\int_{-h}^{-d_\mathrm{s}} \phi^\mathrm{t}_\mathrm{h}(a,z) Z^\mathrm{s}_{k}(z) \mathrm{d}z \\ =\int_{-h}^{-d_\mathrm{s}}\phi^\mathrm{t}_\mathrm{p}(a,z) Z^\mathrm{s}_{k}(z) \mathrm{d}z-\int_{-h}^{-d_\mathrm{s}}\phi^\mathrm{s}_\mathrm{p}(a,z) Z^\mathrm{s}_{k}(z) \mathrm{d}z
\end{multline}
and
\begin{multline}\label{separated velocity matching integral}
    \int_{-h}^{-d_\mathrm{s}} \frac{\partial \phi^\mathrm{s}_\mathrm{h}}{\partial r}(a,z)  Z^\mathrm{t}_{k}(z) \mathrm{d}z-\int_{-h}^{-d_\mathrm{t}} \frac{\partial \phi_\mathrm{h}^\mathrm{t}}{\partial r}(a,z) Z^\mathrm{t}_{k}(z) \mathrm{d}z \\ =\int_{-h}^{-d_\mathrm{t}}\frac{\partial \phi_\mathrm{p}^\mathrm{t}}{\partial r}(a,z) Z^\mathrm{t}_{k}(z) \mathrm{d}z-\int_{-h}^{-d_\mathrm{s}}\frac{\partial \phi_\mathrm{p}^\mathrm{s}}{\partial r}(a,z) Z^\mathrm{t}_{k}(z) \mathrm{d}z
\end{multline}
The right-hand-side of Eq.~\ref{separated potential matching integral} and~\ref{separated velocity matching integral} are known, while the left-hand-side contains unknowns. After substituting for the homogeneous potentials, using orthogonality of the vertical eigen functions, and rearranging, Eq.~\ref{separated potential matching integral} and~\ref{separated velocity matching integral} become
\begin{multline}\label{separated potential matching integral simplified}
    (h-d_\mathrm{s}) \left( C^\mathrm{s}_{1k} R_{1k}^\mathrm{s} (a)+ C^\mathrm{s}_{2k} R_{2k}^\mathrm{s}(a) \right) \\ -\sum _{n = 1}^{N^\mathrm{t}} \left(C^\mathrm{t}_{1n} R_{1n}^\mathrm{t}(a)+ C^\mathrm{t}_{2n} R_{2n}^\mathrm{t}(a) \right)\int_{-h}^{-d_\mathrm{s}} Z^\mathrm{s}_{k}(z) Z_{n}^\mathrm{t}(z)\mathrm{d}z \\ =\int_{-h}^{-d_\mathrm{s}}\left( \phi_\mathrm{p}^\mathrm{t}(a,z) - \phi_\mathrm{p}^\mathrm{s}(a,z) \right) Z^\mathrm{s}_{k}(z) \mathrm{d}z
\end{multline}
and
\begin{multline}\label{separated velocity matching integral simplified}
    \sum _{n = 1}^{N^\mathrm{s}} \left(C^\mathrm{s}_{1n} \frac{\partial R_{1n}^\mathrm{s}}{\partial r}(a)+ C^\mathrm{s}_{2n} \frac{\partial R_{2n}^\mathrm{s}}{\partial r}(a) \right)\int_{-h}^{-d_\mathrm{s}} Z^\mathrm{t}_{k}(z) Z_{n}^\mathrm{s}(z)  \mathrm{d}z \\ - (h-d_\mathrm{t}) \left( C^\mathrm{t}_{1k} \frac{\partial R_{1k}^\mathrm{t}}{\partial r} (a)+ C^\mathrm{t}_{2k} \frac{\partial R_{2k}^\mathrm{t}}{\partial r}(a) \right) \\ = \int_{-h}^{-d_\mathrm{t}}\frac{\partial \phi_\mathrm{p}^\mathrm{t}}{\partial r}(a,z) Z^\mathrm{t}_{k}(z) \mathrm{d}z-\int_{-h}^{-d_\mathrm{s}}\frac{\partial \phi_\mathrm{p}^\mathrm{s}}{\partial r}(a,z) Z^\mathrm{t}_{k}(z) \mathrm{d}z.
\end{multline}
Applying Eq.~\ref{separated potential matching integral simplified} for $k=1,2,...,N^\mathrm{s}$ and Eq.~\ref{separated velocity matching integral simplified} for $k=1,2,...,N^\mathrm{t}$ yields
\begin{multline}\label{potential matching vector equations}
    (h-d_\mathrm{s})\mathrm{diag}\left( \vec{R}_{1}^\mathrm{s}(a)\right) \vec{C}_{1}^\mathrm{s}  + (h-d_\mathrm{s})\mathrm{diag}\left(\vec{R}_{2}^\mathrm{s}(a)\right) \vec{C}_{2}^\mathrm{s} \\ - \boldsymbol{\mathcal{Z}}^{\mathrm{st}} \odot \boldsymbol{1}_{N^\mathrm{s}1}\vec{R}_{1}^\mathrm{t}(a)\vec{C}_1^\mathrm{t}  - \boldsymbol{\mathcal{Z}}^{\mathrm{st}} \odot \boldsymbol{1}_{N^\mathrm{s}1}\vec{R}_{2}^\mathrm{t}(a)\vec{C}_2^\mathrm{t} \\ =\int_{-h}^{-d_\mathrm{s}}\left( \phi_\mathrm{p}^\mathrm{t}(a,z) - \phi_\mathrm{p}^\mathrm{s}(a,z) \right) \vec{Z}^\mathrm{s}(z) \mathrm{d}z
\end{multline}
and
\begin{multline}\label{velocity matching vector equations}
    \boldsymbol{\mathcal{Z}}^\mathrm{ts} \odot \mathbf{1}_{N^\mathrm{t}1}\vec{R}_1^\mathrm{s}(a)\vec{C}_1^\mathrm{s}+\boldsymbol{\mathcal{Z}}^\mathrm{ts} \odot \mathbf{1}_{N^\mathrm{t}1}\vec{R}_2^\mathrm{s}(a)\vec{C}_2^\mathrm{s}\\ -(h-d_\mathrm{t})\mathrm{diag}\left( \frac{\partial}{\partial r} \vec{R}_1^\mathrm{t}(a)\right)\vec{C}_1^\mathrm{t} -(h-d_\mathrm{t})\mathrm{diag}\left( \frac{\partial}{\partial r} \vec{R}_2^\mathrm{t}(a)\right)\vec{C}_2^\mathrm{t} \\ = \int_{-h}^{-d_\mathrm{t}}\frac{\partial \phi_\mathrm{p}^\mathrm{t}}{\partial r}(a,z) \vec{Z}^\mathrm{t}{}(z) \mathrm{d}z-\int_{-h}^{-d_\mathrm{s}}\frac{\partial \phi_\mathrm{p}^\mathrm{s}}{\partial r}(a,z) \vec{Z}^\mathrm{t}{}(z) \mathrm{d}z.
\end{multline}
where $\vec{R}_1^\mathrm{s}(r)=[R_{11}^\mathrm{s}(r), R_{12}^\mathrm{s}(r),...,R_{1N^\mathrm{s}}^\mathrm{s}(r) ]$, $\vec{R}_2^\mathrm{s}(r)=[R_{21}^\mathrm{s}(r), R_{22}^\mathrm{s}(r),...,R_{2N^\mathrm{s}}^\mathrm{s}(r) ]$, $\vec{R}_1^\mathrm{t}(r)=[R_{11}^\mathrm{t}(r), R_{12}^\mathrm{t}(r),...,R_{1N^\mathrm{t}}^\mathrm{t}(r) ]$, and $\vec{R}_2^\mathrm{t}(r)=[R_{21}^\mathrm{t}(r), R_{22}^\mathrm{t}(r),...,R_{2N^\mathrm{t}}^\mathrm{t}(r) ]$ are row vectors of radial eigen-functions; $\vec{C}_1^\mathrm{s}=[C_{11}^\mathrm{s}, C_{12}^\mathrm{s},...,C_{1N^\mathrm{s}}^\mathrm{s}]^T$, $\vec{C}_2^\mathrm{s}=[C_{21}^\mathrm{s}, C_{22}^\mathrm{s},...,C_{2N^\mathrm{s}}^\mathrm{s}]^T$, $\vec{C}_1^\mathrm{t}=[C_{11}^\mathrm{t}, C_{12}^\mathrm{t},...,C_{1N^\mathrm{t}}^\mathrm{t}]^T$, and $\vec{C}_2^\mathrm{t}=[C_{21}^\mathrm{t}, C_{22}^\mathrm{t},...,C_{2N^\mathrm{t}}^\mathrm{t}]^T$ are column vectors of eigen-coefficients; $\vec{Z}^\mathrm{s}(z)=[Z_1^\mathrm{s}(z),Z_2^\mathrm{s}(z),...,Z_{N^\mathrm{s}}^\mathrm{s}(z)]^T$ and $\vec{Z}^\mathrm{t}(z)=[Z_1^\mathrm{t}(z),Z_2^\mathrm{t}(z),...,Z_{N^\mathrm{t}}^\mathrm{t}(z)]^T$ are column vectors of vertical eigen-functions; $\boldsymbol{\mathcal{Z}}^\mathrm{st}=\boldsymbol{\mathcal{Z}}^\mathrm{ts}{}^T=\int_{-h}^{-d_\mathrm{s}} \vec{Z}^\mathrm{s}{}(z) \otimes \vec{Z}^\mathrm{t}(z) \mathrm{d}z$ is an $N^\mathrm{s} \times N^\mathrm{t}$ matrix of coupling integrals; $\mathbf{1}_{N^\mathrm{s}1}$ and $\mathbf{1}_{N^\mathrm{t}1}$ are $N^\mathrm{s} \times 1$ and $N^\mathrm{t} \times 1$ matrices of ones, respectively; and $\odot$ is the Hadamard (element-wise) product. Eq.~\ref{potential matching vector equations} and~\ref{velocity matching vector equations} contain $N^\mathrm{s}$ and $N^\mathrm{t}$ equations, respectively. However, they have $2N^\mathrm{s} + 2N^\mathrm{t}$ unknown eigen-coefficients. Recall that these equations correspond to enforcing conditions at a single boundary between fluid regions, as shown in Fig.~\ref{fig:Matching Diagram}. Section~\ref{Block Matrix Structure} shows how to organize Eq.~\ref{potential matching vector equations} and~\ref{velocity matching vector equations} for the $M$ boundaries so that all eigen-coefficients can be solved for simultaneously.

Note that all coupling integrals have closed form solutions. The element in the $n$th row and $k$th column of the coupling integral matrix associated with matching between region $i_m$ and $i_{m+1}$ is

\begin{equation}\label{coupling integral i_m and i_m+1}
    [\boldsymbol{\mathcal{Z}}^{\text{s} \text{t}} ]_{nk}=-\frac{\sin ((d_\text{s}-h) (\lambda_n^{\text{s}}-\lambda_k^{\text{t}}))}{\lambda_n^{\text{s}}-\lambda_k^{\text{t}}}-\frac{\sin ((d_\text{s}-h) (\lambda_n^{\text{s}}+\lambda_k^{\text{t}}))}{\lambda_n^{\text{s}}+\lambda_k^{\text{t}}}
\end{equation}

\noindent where the superscripts and subscripts associated with the shorter and taller fluid are assigned to $i_m$ or $i_{m+1}$ depending on if $d_m>d_{m+1}$ or $d_{m+1}>d_m$. Eq.~\ref{coupling integral i_m and i_m+1} holds for $1 \le m \le M-1$. For matching at the interface between the outermost inner region $i_M$ and the external region $e$, a different expression is used. The element in the $n$th row and $k$th column of the coupling integral matrix associated with matching between region $i_M$ and $e$ is

\begin{equation}\label{coupling integral i_M and e}
    [\boldsymbol{\mathcal{Z}}^{i_M e} ]_{nk}=-\frac{(\lambda_k^{e}+\lambda_n^{i_M}) \sin ((d_M-h) (\lambda_k^{e}-\lambda_n^{i_M}))+(\lambda_k^{e}-\lambda_n^{i_M}) \sin ((d_M-h) (\lambda_k^{e}+\lambda_n^{i_M}))}{(\lambda_k^{e}-\lambda_n^{i_M}) (\lambda_k^{e}+\lambda_n^{i_M}) \sqrt{\frac{\sin (2 h \lambda_k^{e})}{2 h \lambda_k^{e}}+1}}.
\end{equation}
Note that Eq.~\ref{coupling integral i_m and i_m+1} and~\ref{coupling integral i_M and e} can simplify considerably when $n$ or $k$ are zero. 

% If we write Eq.~\ref{separated potential matching integral simplified} for $k=1,2,...,N^\mathrm{s}$ and Eq.~\ref{separated velocity matching integral simplified} for $k=1,2,...,N^\mathrm{t}$, this gives a total of $N^\mathrm{t} + N^\mathrm{s}$ equations with $N^\mathrm{t} + N^\mathrm{s}$ unknowns. To apply this matching process to $M$ heaving surface-piercing compound annular cylinders, we can consider three different matching cases: 1) the $r=a_1$ boundary, 2) the $r=a_m$ boundary for $1<m<M$, and 3) the $r=a_M$ boundary. For the $r=a_1$ boundary, the potential in region $i_1$ has $N^{i_1}$ unknowns while the potential in the $i_2$ region has $2N^{i_2}$ unknowns. Eq.~\ref{separated potential matching integral simplified} and~\ref{separated velocity matching integral simplified} can be applied by assigning the taller and smaller regions to $i_1$ or $i_2$, setting the second eigen-coefficient (either $C_{2n}^\mathrm{s}$ or $C_{2n}^\mathrm{t}$) equal to zero for the $i_1$ region, and substituting the eigen-functions from Table~\ref{tab:MEEM-eigen-functions} into the equations. For the $r=a_m$ boundary for $1<m<M$, the potential in region $i_m$ has $2N^{i_m}$ unknowns while the potential in the $i_{m+1}$ region has $2N^{i_{m+1}}$ unknowns. Eq.~\ref{separated potential matching integral simplified} and~\ref{separated velocity matching integral simplified} can be applied by assigning the taller and smaller regions to $i_1$ or $i_2$ and substituting the eigen-functions from Table~\ref{tab:MEEM-eigen-functions} into the equations. For the $r=a_M$ boundary, the potential in region $i_M$ has $2N^{i_M}$ unknowns while the potential in the $e$ region has $N^{e}$ unknowns. Eq.~\ref{separated potential matching integral simplified} and~\ref{separated velocity matching integral simplified} can be applied by assigning the shorter region to $i_M$ and the taller region to $e$, setting $d_\mathrm{t}=0$ and $C_{2n}^t=0$, and substituting the eigen-functions from Table~\ref{tab:MEEM-eigen-functions} into the equations. Applying these equations for every boundary will result in a total of $N_\mathrm{T}$ equations and $N_\mathrm{T}$ unknowns.

\subsection{Block Matrix Structure \textcolor{blue}{Collin finish rereading}}\label{Block Matrix Structure}
Rewriting Eq.~\ref{potential matching vector equations} and~\ref{velocity matching vector equations} for each of the $M$ boundaries, where the symbols representing the smaller $\mathrm{s}$ and taller $\mathrm{t}$ regions are replaced by the region names of the problem (i.e. $i_1, i_2,..., i_M,e$) and $a$ is replaced by $a_m$, yields a set of $N_\mathrm{T}$ equations that are linear with respect to the unknown eigen-coefficients. The structure of these equations is in the form of $\mathbf{A} \vec{x}=\vec{b}$ where the matrix $\mathbf{A} \in \mathbb{C}^{N_\mathrm{T}\times N_\mathrm{T}}$ contains the left-hand-side of Eq.~\ref{potential matching vector equations} and~\ref{velocity matching vector equations} (excluding the eigen-coefficients), the vector $\vec{x}=[\vec{C}_{1}^{i_1}, \vec{C}_{1}^{i_2}, \vec{C}_{2}^{i_2},...,\vec{C}_{1}^{i_M}, \vec{C}_{2}^{i_M}, \vec{C}_{1}^{e}]^T \in \mathbb{C}^{N_\mathrm{T}}$ contains all eigen-coefficients, and the vector $\vec{b} \in \mathbb{R}^{N_\mathrm{T}}$ contains the right-hand-side of Eq.~\ref{potential matching vector equations} and~\ref{velocity matching vector equations}. As shown in Table~\ref{tab:MEEM-A-matrix}, the $\mathbf{A}$ is a block diagonal matrix composed of matrices $\mathbf{A}_1,\mathbf{A}_2,...,\mathbf{A}_M$ and zero matrices. Each matrix $\mathbf{A}_m$ contains the left-hand-sides of Eq.~\ref{potential matching vector equations} and~\ref{velocity matching vector equations} (excluding the eigen-coefficients) when applying them to the $m$th boundary. The corresponding right-hand-sides of Eq.~\ref{potential matching vector equations} and~\ref{velocity matching vector equations} are contained in $\vec{b}_m$, which form the vector $\vec{b}=[\vec{b}_1,\vec{b}_2,...,\vec{b}_M]^T$

Depending on which fluid region is the taller, there are two cases of $\mathbf{A}_m$: 1) when $d_m>d_{m+1}$, which is shown in Table~\ref{tab:MEEM-A_m-matrix-case-1},  and 2) when $d_m<d_{m+1}$, which is shown in Table~\ref{tab:MEEM-A_m-matrix-case-2}. The corresponding right-hand-side vectors $\vec{b}_m$ are shown in Table~\ref{tab:MEEM-b_m-vector-case-1} and~\ref{tab:MEEM-b_m-vector-case-2}, respectively. These equations are valid as is when $2 \le m \le M$. When $m=1$, the $\vec{C}_2^{i_1}$ column should be excluded from $\mathbf{A}_1$ since the only unknowns associated with $\phi^{i_1}$ are $\vec{C}_1^{i_1}$. Similarly, when $m=M$, all indices with $i_{m+1}$ should be replaced with $e$, $d_{m+1}$ should be set to zero, and the $\vec{C}_2^{i_{m+1}}$ column should be excluded from $\mathbf{A}_M$. Consequently, $A_m \in \mathbb{R}^{(N^{i_m} + N^{i_{m+1}}) \times (2N^{i_m} + 2N^{i_{m+1}})}$, while $A_1 \in \mathbb{R}^{(N^{i_1} + N^{i_{2}}) \times (N^{i_1} + 2N^{i_2})}$ and $A_M \in \mathbb{R}^{(N^{i_M} + N^{e}) \times (2N^{i_M} + N^e)}$. This is consistent with what is shown in Table~\ref{tab:MEEM-A-matrix}. Tables~\ref{tab:MEEM-b_m-vector-case-1} and~\ref{tab:MEEM-b_m-vector-case-2} can be used without modification to obtain all the vectors $\vec{b}_m$ for $1 \le m \le M-1$ while, for $m=M$, Table~\ref{tab:MEEM-b_m-vector-case-1} should be modified so that all indices with $i_{m+1}$ are replaced with $e$ and $d_{m+1}$ is set to zero.     






% \begin{landscape}
% \begin{table}
%     \centering
%     \begin{tabular}{|>{\centering\arraybackslash}p{0.18\linewidth}|c||c|c|c|c|c|c|c|c|c|c|c|c|}
%     \hline
%      & & $\vec{C}_{1n}^{i_1}$& $\vec{C}_{1n}^{i_2}$& $\vec{C}_{2n}^{i_2}$ & $\vec{C}_{1n}^{i_3}$& $\vec{C}_{2n}^{i_3}$ & ... 
%      &$\vec{C}_{1n}^{i_{M-1}}$& $\vec{C}_{2n}^{i_{M-1}}$ &$\vec{C}_{1n}^{i_M}$& $\vec{C}_{2n}^{i_M}$ & $\vec{C}_n^e$ \\\hline 
%       &size&  $N^{i_1}$&  $N^{i_2}$&  $N^{i_2}$& $N^{i_3}$&  $N^{i_3}$ &... & $N^{i_{M-1}}$&  $N^{i_{M-1}}$ & $N^{i_M}$ & $N^{i_M}$ & $N^e$\\ \hline \hline 
      
%       % A1 block
%       \shortstack{$\phi^{i_1}=\phi^{i_2}$ \\ at $r=a_1$}
%         & \multirow{2}{*}{$N^{i_1} + N^{i_2}$} 
%         & \multicolumn{3}{c|}{\multirow{2}{*}{$\mathbf{A}_1$}} 
%         & & & & & & & & \\ \cline{1-1}\cline{6-13}
%       \shortstack{$\frac{\partial}{\partial r}\phi^{i_1}=\frac{\partial}{\partial r}\phi^{i_2}$ \\ at $r=a_1$}
%         & & \multicolumn{3}{c|}{} 
%         & & & & & & & & \\ \hline 
      
%       % A2 block
%       \shortstack{$\phi^{i_2}=\phi^{i_3}$ \\ at $r=a_2$}
%         & \multirow{2}{*}{$N^{i_2}+N^{i_3}$} 
%         & & \multicolumn{4}{c|}{\multirow{2}{*}{$\mathbf{A}_2$}} 
%         & & & & & & \\ \cline{1-1}\cline{3-3}\cline{8-13}
%       \shortstack{$\frac{\partial}{\partial r}\phi^{i_2}=\frac{\partial}{\partial r}\phi^{i_3}$ \\ at $r=a_2$}
%         & & & \multicolumn{4}{c|}{} 
%         & & & & & & \\ \hline 
      
%         $\vdots$ & $\vdots$
%             & \multicolumn{2}{c|}{$\mathbf{0}$} % empty columns to left of diagonal
%             & \multicolumn{7}{c|}{$\ddots$} % the diagonal itself
%             & \multicolumn{2}{c|}{$\mathbf{0}$} % empty columns to right of diagonal
%              \\ \hline

      
%       % A_{M-1} block
%       \shortstack{$\phi^{i_{M-1}}=\phi^{i_M}$ \\ at $r=a_{M-1}$}
%         & \multirow{2}{*}{$N^{i_{M-1}} + N^{i_M}$} 
%         & & & & & & & \multicolumn{4}{c|}{\multirow{2}{*}{$\mathbf{A}_{M-1}$}} 
%         & \\ \cline{1-1}\cline{3-8}\cline{13-13}
%       \shortstack{$\frac{\partial}{\partial r}\phi^{i_{M-1}}=\frac{\partial}{\partial r}\phi^{i_M}$ \\ at $r=a_{M-1}$}
%         & & & & & & & & \multicolumn{4}{c|}{} 
%         & \\ \hline
      
%       % A_M block
%       \shortstack{$\phi^{i_{M}}=\phi^{e}$ \\ at $r=a_M$}
%         & \multirow{2}{*}{$N^{i_M}+N^e$} 
%         & & & & & & & & & \multicolumn{3}{c|}{\multirow{2}{*}{$\mathbf{A}_M$}} \\ \cline{1-1}\cline{3-10}
%       \shortstack{$\frac{\partial}{\partial r}\phi^{i_M}=\frac{\partial}{\partial r}\phi^{e}$ \\ at $r=a_M$}
%         & & & & & & & & & & \multicolumn{3}{c|}{} \\ \hline
%     \end{tabular}
%     \caption{MEEM A-matrix}
%     \label{tab:MEEM-A-matrix}
% \end{table}
% \end{landscape}



\begin{table}
    \centering
    \begin{tabular}{|>{\centering\arraybackslash}p{0.18\linewidth}|c||c|c|c|c|c|c|c|c|c|c|c|c|}
    \hline
     & & $\vec{C}_{1}^{i_1}$& $\vec{C}_{1}^{i_2}$& $\vec{C}_{2}^{i_2}$ & $\vec{C}_{1}^{i_3}$& $\vec{C}_{2}^{i_3}$ & ... 
     &$\vec{C}_{1}^{i_{M-1}}$& $\vec{C}_{2}^{i_{M-1}}$ &$\vec{C}_{1}^{i_M}$& $\vec{C}_{2}^{i_M}$ & $\vec{C}_1^e$ \\\hline 
      &size&  $N^{i_1}$&  $N^{i_2}$&  $N^{i_2}$& $N^{i_3}$&  $N^{i_3}$ &... & $N^{i_{M-1}}$&  $N^{i_{M-1}}$ & $N^{i_M}$ & $N^{i_M}$ & $N^e$\\ \hline \hline 
      
      % Row 1
      Boundary 1 & $N^{i_1}+N^{i_2}$ 
        & \multicolumn{3}{c|}{$\mathbf{A}_1$} &\multicolumn{8}{c|}{$\mathbf{0}$} \\ \hline
      
      % Row 2
      Boundary 2 & $N^{i_2}+N^{i_3}$
        & $\mathbf{0}$ & \multicolumn{4}{c|}{$\mathbf{A}_2$} &\multicolumn{6}{c|}{$\mathbf{0}$} \\ \hline

      % Row 3
      $\vdots$ & $\vdots$
        & \multicolumn{11}{c|}{$\ddots$} \\ \hline

      % Row 4
      Boundary $M-1$& $N^{i_{M-1}}+N^{i_M}$
        & \multicolumn{6}{c|}{$\mathbf{0}$} & \multicolumn{4}{c|}{$\mathbf{A}_{M-1}$} & $\mathbf{0}$ \\ \hline

      % Row 5
      Boundary $M$ & $N^{i_M}+N^{i_e}$
        & \multicolumn{8}{c|}{$\mathbf{0}$} & \multicolumn{3}{c|}{$\mathbf{A}_M$} \\ \hline
    \end{tabular}
    \caption{MEEM A-matrix.}
    \label{tab:MEEM-A-matrix}
\end{table}


\begin{landscape}
\begin{table}
    \centering
    \begin{tabular}{|>{\centering\arraybackslash}p{0.18\linewidth}|c||c|c|c|c|c|}
    \hline
     & & $\vec{C}_{1}^{i_m}$& $\vec{C}_{2}^{i_m}$& $\vec{C}_{1}^{i_{m+1}}$ & $\vec{C}_{2}^{i_{m+1}}$ \\\hline 
      &size&  $N^{i_m}$&  $N^{i_m}$&  $N^{i_{m+1}}$& $N^{i_{m+1}}$\\ \hline \hline 
      
      % Row 1
      \shortstack{$\phi^{i_m}=\phi^{i_{m+1}}$ \\ at $r=a_m$} & $N^{i_m}$ & $(h-d_m) \mathrm{diag}\left( \vec{R}_1^{i_m}\right)$ & $(h-d_m) \mathrm{diag}\left( \vec{R}_2^{i_m}\right)$ & $-\boldsymbol{\mathcal{Z}}^{i_mi_{m+1}} \odot \mathbf{1}_{N^{i_m}1} \vec{R}_1^{i_{m+1}}$ & $-\boldsymbol{\mathcal{Z}}^{i_mi_{m+1}} \odot \mathbf{1}_{N^{i_m}1} \vec{R}_2^{i_{m+1}}$ \\ \hline
      
      % Row 2
      \shortstack{$\frac{\partial}{\partial r}\phi^{i_m}=\frac{\partial}{\partial r}\phi^{i_{m+1}}$ \\ at $r=a_m$} & $N^{i_{m+1}}$
        & $\boldsymbol{\mathcal{Z}}^{i_{m+1}i_m} \odot \mathbf{1}_{N^{i_{m+1}}1} \vec{R}_1^{i_{m}}$ & $\boldsymbol{\mathcal{Z}}^{i_{m+1}i_m} \odot \mathbf{1}_{N^{i_{m+1}}1} \vec{R}_2^{i_{m}}$ & $-(h-d_{m+1}) \mathrm{diag}\left( \frac{\partial}{\partial r} \vec{R}_1^{i_{m+1}}\right)$ & $-(h-d_{m+1}) \mathrm{diag}\left( \frac{\partial}{\partial r} \vec{R}_2^{i_{m+1}}\right)$ \\ \hline
    \end{tabular}
    \caption{MEEM $\mathbf{A}_m$-matrix when $d_m>d_{m+1}$ ($i_m = \mathrm{s}$ and $i_{m+1} = \mathrm{t}$). Note all radial eigen-functions and there derivatives are evaluated at $r=a_m$.}
    \label{tab:MEEM-A_m-matrix-case-1}
\end{table}
\end{landscape}


\begin{landscape}
\begin{table}
    \centering
    \begin{tabular}{|>{\centering\arraybackslash}p{0.18\linewidth}|c||c|c|c|c|c|}
    \hline
     & & $\vec{C}_{1}^{i_m}$& $\vec{C}_{2}^{i_m}$& $\vec{C}_{1}^{i_{m+1}}$ & $\vec{C}_{2}^{i_{m+1}}$ \\\hline 
      &size&  $N^{i_m}$&  $N^{i_m}$&  $N^{i_{m+1}}$& $N^{i_{m+1}}$\\ \hline \hline 
      
      % Row 1
      \shortstack{$\phi^{i_m}=\phi^{i_{m+1}}$ \\ at $r=a_m$} & $N^{i_{m+1}}$ & $-\boldsymbol{\mathcal{Z}}^{i_{m+1}i_m} \odot \mathbf{1}_{N^{i_{m+1}}1} \vec{R}_1^{i_{m}}$ & $-\boldsymbol{\mathcal{Z}}^{i_{m+1}i_m} \odot \mathbf{1}_{N^{i_{m+1}}1} \vec{R}_2^{i_{m}}$ & $(h-d_{m+1}) \mathrm{diag}\left( \vec{R}_1^{i_{m+1}}\right)$ & $(h-d_{m+1}) \mathrm{diag}\left( \vec{R}_2^{i_{m+1}}\right)$ \\ \hline
      
      % Row 2
      \shortstack{$\frac{\partial}{\partial r}\phi^{i_m}=\frac{\partial}{\partial r}\phi^{i_{m+1}}$ \\ at $r=a_m$} & $N^{i_{m}}$
        & $-(h-d_{m}) \mathrm{diag}\left( \frac{\partial}{\partial r} \vec{R}_1^{i_{m}}\right)$ & $-(h-d_{m}) \mathrm{diag}\left( \frac{\partial}{\partial r} \vec{R}_2^{i_{m}}\right)$ & $\boldsymbol{\mathcal{Z}}^{i_mi_{m+1}} \odot \mathbf{1}_{N^{i_{m}}1} \vec{R}_1^{i_{m+1}}$ & $\boldsymbol{\mathcal{Z}}^{i_mi_{m+1}} \odot \mathbf{1}_{N^{i_{m}}1} \vec{R}_2^{i_{m+1}}$ \\ \hline
    \end{tabular}
    \caption{MEEM $\mathbf{A}_m$-matrix when $d_m<d_{m+1}$ ($i_m = \mathrm{t}$ and $i_{m+1} = \mathrm{s}$). Note all radial eigen-functions and there derivatives are evaluated at $r=a_m$.}
    \label{tab:MEEM-A_m-matrix-case-2}
\end{table}
\end{landscape}


\begin{table}
    \centering
    \begin{tabular}{|>{\centering\arraybackslash}p{0.18\linewidth}|c||c|c|}
      \hline
      &size& \\ \hline \hline 
      
      % Row 1
      \shortstack{$\phi^{i_m}=\phi^{i_{m+1}}$ \\ at $r=a_m$} & $N^{i_m}$ & $\int_{-h}^{-d_m}\left( \phi_\mathrm{p}^{i_{m+1}} - \phi_\mathrm{p}^{i_m} \right) \vec{Z}^{i_m} \mathrm{d}z$\\ \hline
      
      % Row 2
      \shortstack{$\frac{\partial}{\partial r}\phi^{i_m}=\frac{\partial}{\partial r}\phi^{i_{m+1}}$ \\ at $r=a_m$} & $N^{i_{m+1}}$
        & $\int_{-h}^{-d_{m+1}}\frac{\partial \phi_\mathrm{p}^{i_{m+1}}}{\partial r} \vec{Z}^{i_{m+1}} \mathrm{d}z-\int_{-h}^{-d_m}\frac{\partial \phi_\mathrm{p}^{{i_{m}}}}{\partial r} \vec{Z}^{i_{m+1}}{} \mathrm{d}z$ \\ \hline
    \end{tabular}
    \caption{MEEM $\vec{b}_m$ vector when $d_m>d_{m+1}$ ($i_m = \mathrm{s}$ and $i_{m+1} = \mathrm{t}$). Note all radial eigen-functions and their derivatives are evaluated at $r=a_m$.}
    \label{tab:MEEM-b_m-vector-case-1}
\end{table}


\begin{table}
    \centering
    \begin{tabular}{|>{\centering\arraybackslash}p{0.18\linewidth}|c||c|c|}
    \hline 
      &size& \\ \hline \hline 
      
      % Row 1
      \shortstack{$\phi^{i_m}=\phi^{i_{m+1}}$ \\ at $r=a_m$} & $N^{i_{m+1}}$ & $\int_{-h}^{-d_{m+1}}\left( \phi_\mathrm{p}^{i_{m}} - \phi_\mathrm{p}^{i_{m+1}} \right) \vec{Z}^{i_{m+1}} \mathrm{d}z$\\ \hline
      
      % Row 2
      \shortstack{$\frac{\partial}{\partial r}\phi^{i_m}=\frac{\partial}{\partial r}\phi^{i_{m+1}}$ \\ at $r=a_m$} & $N^{i_{m}}$
        &  $\int_{-h}^{-d_m}\frac{\partial \phi_\mathrm{p}^{i_{m}}}{\partial r} \vec{Z}^{i_{m}} \mathrm{d}z-\int_{-h}^{-d_{m+1}}\frac{\partial \phi_\mathrm{p}^{i_{m+1}}}{\partial r} \vec{Z}^{{i_{m}}} \mathrm{d}z$\\ \hline
    \end{tabular}
    \caption{MEEM $\vec{b}_m$ vector when $d_m>d_{m+1}$ ($i_{m+1} = \mathrm{s}$ and $i_{m} = \mathrm{t}$). Note all radial eigen-functions and their derivatives are evaluated at $r=a_m$.}
    \label{tab:MEEM-b_m-vector-case-2}
\end{table}




\subsection{Hydrodynamic and Hydrostatic Forces}\label{Hydrodynamic Forces}
In this section, we will characterize the radiation, excitation, and hydrostatic forces of a system with $M$ internal regions and $Q$ heave degrees of freedom. Note that all forces are applied in the $\hat{e}_z$ direction. Also, while there are $M$ internal regions, multiple regions may form a single body if the regions are rigidly fixed to one another. Thus, for a system with a total of $M$ internal regions and $Q$ heave DOFs, $Q \le M$. 

Moving forward, $\mathbf{A} \vec{x}_q = \vec{b}_q$ will indicate the system of equations associated with the motion of only the $q$th body (and DOF) of the system, while all other bodies are fixed. Note that the matrix $\mathbf{A}$ does not depend on which body is moving. Meanwhile, $\vec{b}_q$, and hence the solution $\vec{x}_q$, does. This is due to $\vec{b}_q$ containing integrals of particular potentials, which are zero for stationary regions and, in general, non-zero for moving regions.

First, we will find the total radiation force on the $p$th body due to the $q$th DOF. In the time domain, this is 
\begin{equation}
    \vec{f}_{pq}(t) = \iint_{S_p}P_q(t) \hat{n}_p dS = - \rho \iint_{S_p}\frac{\partial  \ _{}^{q}\Phi(\mathbf{x},t)}{\partial t} \hat{n}_p dS
\end{equation}
where $S_p$ is the wetted surface of body $p$, $P_q(t)$ is the pressure in the fluid due to motion of the $q$th DOF, $\hat{n}_p$ is the unit vector normal to the wetted surface of body $p$ pointing outward from the fluid, $\rho$ is the density of the fluid, and $_{}^{q}\Phi(\mathbf{x},t)$ is the potential due to motion of the $q$th DOF. This can be rewritten in the frequency domain as 
\begin{equation}\label{eq:freq domain vector of rad force}
    \vec{\hat{f}}_{pq} = \text{i} \omega  \rho \iint_{S_p} \ _{}^{q}\phi(r,z) \ \hat{n}_p dS
\end{equation}
where $\vec{f}_{pq}(t) = \mathrm{Re} \{ \vec{\hat{f}}_{pq} e^{- \text{i} \omega t} \}$ and $_{}^{q}\Phi(\mathbf{x},t) = \mathrm{Re} \{ _{}^{q}\phi(r,z) e^{- \text{i} \omega t} \}$. Dotting Eq.~\ref{eq:freq domain vector of rad force} with the unit vector $\hat{e}_z$ yields the complex heave force on body $p$ due to the motion of body $q$ 
\begin{equation}\label{eq:freq domain scalar rad force}
    \hat{f}_{pq} = \vec{\hat{f}}_{pq} \cdot \hat{e}_z = \text{i} \omega \rho \iint_{S_p} \ _{}^{q}\phi(r,z) \ (\hat{n}_p \cdot \hat{e}_z ) \  dS.
\end{equation}
Since $\hat{n}_p=\hat{e}_z$ at the horizontal portions of $S_p$ and $\hat{n}_p=\hat{e}_r$ at any vertical portions of $S_p$, integration on only the bottom surface of each region contributes to forces in heave. Note that this will not be the case in Sec.~\ref{sec:slant} when we use a finite number of cylindrical regions to approximate an axisymmetric body with a slanted surface that is not purely vertical or horizontal. Proceeding with integration along the bottom body boundaries, the total heave force on body $p$ will be due to integrating the potential on the bottom boundaries of all regions belonging to body $p$. We will define $\mathcal{M}_p$ as the set of all indices that correspond to regions which form the $p$th body. For example, if body 1 consists of regions $i_1$, $i_3$ and $i_4$, $\mathcal{M}_1 = \{1, 3, 4\}$. Eq.~\ref{eq:freq domain scalar rad force} can be written in terms of the potential at each region by
\begin{equation}\label{eq:freq domain scalar rad force in terms of regions}
    \hat{f}_{pq} = \text{i} \omega \rho \sum_{m \in \mathcal{M}_p}\int_0^{2 \pi} \int_{a_m}^{a_{m+1}} \ _{}^{q}\phi^{i_m}(r,-d_m) \ r  \ dr\ d\theta
\end{equation}
where $_{}^{q}\phi^{i_m}(r,-d_m)$ is the potential in internal region $i_m$ evaluated at $z=-d_m$ when only the $q$th DOF is moving. $\hat{f}_{pq}$ can be rewritten in terms of frequency-dependent added mass and radiation damping coefficients $A_{pq}(\omega)$ and $B_{pq}(\omega)$, respectively. To do this, we will define the velocity of the $q$th body as $v_q(t) = \mathrm{Re} \{ \hat{v}_q e^{- \text{i} \omega t}\}$ so that $\hat{f}_{pq} = (\text{i} \omega A_{pq}(\omega) - B_{pq}(\omega)) \hat{v}_q$. Since $\hat{f}_{pq}(\omega)$ scales proportionally with $\hat{v}_q(\omega)$, $\hat{f}_{pq}(\omega)$ can be determined for any $\hat{v}_q(\omega)$ once $A_{pq}(\omega)$ and $B_{pq}(\omega)$ are known. Thus, there is no loss of generality when prescribing the value of $\hat{v}_q(\omega)$ in the linear potential flow problem to solve for $A_{pq}(\omega)$ and $B_{pq}(\omega)$. Recall from Eq.~\ref{No flux BC at wetted surface} that, if a region is heaving, the particular potentials in Table \ref{tab:MEEM-eigen-functions} were defined so that it does so with unit amplitude velocity. Thus, if regions consisting of the $q$th body are heaving, $\hat{v}_q=1$. Eq.~\ref{eq:freq domain scalar rad force in terms of regions} can be simplified and rearranged as
\begin{equation}\label{eq:freq domain scalar rad force in tyerms of assed mass and damping}
\begin{aligned}
    A_{pq}(\omega) + \frac{\text{i}B_{pq}(\omega)}{\omega} &= 2 \pi \rho \sum_{m \in \mathcal{M}_p} \int_{a_m}^{a_{m+1}} \ _{}^{q}\phi^{i_m}(r,-d_m) \ r  \ dr \\
    &=2 \pi \rho \sum_{m \in \mathcal{M}_p} \Bigg[
    \int_{a_{m}}^{a_{{m+1}}} \ _{}^{q}\phi^{i_m}_p(r,-d_m)\, r\, dr \\
    & + \sum_{n=1}^{N_{i_m}} 
    \ _{}^{q}C_{1n}^{i_m} \ Z_{n}^{i_m}(-d_m)
    \int_{a_{m}}^{a_{{m+1}}} \ R_{1n}^{i_m}(r)\, r\, dr \\
    & + \sum_{n=1}^{N_{i_m}} 
    \ _{}^{q}C_{2n}^{i_m} \ Z_{n}^{i_m}(-d_m)
    \int_{a_{m}}^{a_{{m+1}}} \ R_{2n}^{i_m}(r)\, r\, dr
    \Bigg]
\end{aligned}
\end{equation}
where all symbols written with a left superscript $q$ denote symbols specific to solving the radiation problem when only the $q$th body is moving. A more convenient form of this is 
\begin{equation}\label{A_pq B_pq scalar form}
    A_{pq}(\omega) + \frac{\mathrm{i}B_{pq}(\omega)}{\omega}=2\pi \rho (c_{pq} + \vec{c}_p \ \vec{x}_q) =2\pi \rho (c_{pq} + \vec{c}_p \ \mathbf{A}^{-1} \vec{b}_q)
\end{equation}
with output scalar $c_{pq} \in \mathbb{R}$
\begin{equation}
     c_{pq} = \left\{\begin{matrix} \sum_{m \in \mathcal{M}_p } \frac{\left({a^2_{{m+1}}}-{a^2_{m}}\right)\,\left(-{a^2_{m}}-{a^2_{{m+1}}}+4(h-d_m)^2\right)}{16\,\left(h - d_m\right)} & \text{when} \ p=q \\
    0 & \text{when} \ p \neq q
\end{matrix}\right.
\end{equation}
and output row vector $\vec{c}_p =[\vec{c}_p^{ \ i_1}, \ \vec{c}_p^{ \ i_2}, ..., \vec{c}_p^{ \ i_M},  \vec{c}_p^{ \ e}]\in \mathbb{C}^{N_\mathrm{T}}$ where
\begin{equation}\label{eq:c_q_i1_vec}
    \vec{c}_p^{ \ i_1} = \left\{\begin{matrix} \vec{Z}_n^{i_1}(-d_1) \odot\vec{\mathcal{R}}_{1n}^{i_1} & \text{when} \ m \in \mathcal{M}_p \\
    \vec{0}_{N^{i_1}} & \text{when} \ m \notin \mathcal{M}_p
    \end{matrix}\right. ,
\end{equation}
\begin{equation}\label{eq:c_q_im_vec}
    \vec{c}_p^{ \ i_m} = \left\{\begin{matrix} [\vec{Z}_n^{i_m}(-d_m) \odot\vec{\mathcal{R}}_{1n}^{i_m}, \ \vec{Z}_n^{i_m}(-d_m) \odot\vec{\mathcal{R}}_{2n}^{i_m}] & \text{when} \ m \in \mathcal{M}_p \\
    \vec{0}_{2N^{i_m}} & \text{when} \ m \notin \mathcal{M}_p
    \end{matrix}\right. ,
\end{equation}
for $2\le m \le M$, and $\vec{c}_p^{ \ e} = \vec{0}_{N^{e}}$. Note that $\vec{0}_{k}$ denotes a $k$-long row vector of zeros. The vector $\vec{c}_p$ contains non-zero elements that are positioned so they multiply the corresponding eigen-coefficients for the $m$th region in the $\vec{x}_q$ vector in Eq.~\ref{A_pq B_pq scalar form}. Additionally, the $n$th element of the vectors $\vec{\mathcal{R}}_{1}^{i_m}$ and $\vec{\mathcal{R}}_{2}^{i_m}$ are 
\begin{equation}\label{radial integral 1}
\begin{aligned}
    [\vec{\mathcal{R}}_{1}^{i_m}]_n & = \int_{a_{m}}^{a_{{m+1}}} R_{1n}^{i_m}(r)\, r\, dr \\
    & = \left\{\begin{matrix}
    \frac{1}{4} (a_{{m+1}}^2-a_{m}^2) & \text{ for } n=0 \\
 \frac{a_{{m+1}} \mathrm{I}_1(a_{{m+1}} \lambda_n^{i_m})-a_{m} \mathrm{I}_1(a_{m} \lambda_n^{i_m})}{\lambda_n^{i_m}\mathrm{I}_0(a_{{m}}\lambda_n^{i_m})} & \text{ for } n \ge 1
\end{matrix}\right.
\end{aligned}
\end{equation}

\noindent and

\begin{equation}\label{radial integral 2}
\begin{aligned}
    [\vec{\mathcal{R}}_{2}^{i_m}]_n & = \int_{a_{m}}^{a_{{m+1}}} R_{2n}^{i_m}(r)\, r\, dr \\
    & = \left\{\begin{matrix}
    \frac{1}{8} \left(2 a_{{m+1}}^2 \ln \left(\frac{a_{{m+1}}}{a_{m}}\right)-a_{{m+1}}^2+a_{m}^2\right) & \text{ for } n=0 \\
 \frac{a_m \mathrm{K}_1(a_m \lambda^{i_m}_n)-a_{{m+1}} \mathrm{K}_1(a_{{m+1}} \lambda^{i_m}_n)}{\lambda^{i_m}_n \mathrm{K}_0(a_m \lambda^{i_m}_n)} & \text{ for } n \ge 1
\end{matrix}\right.
\end{aligned}
\end{equation}
By applying Eq.~\ref{A_pq B_pq scalar form} for $q=1,2,...,Q$ and $p=1,2,...,Q$, we can express the added mass matrix $\mathbf{A}_\mathrm{r}(\omega) \in \mathbb{R}^{Q \times Q}$ and radiation damping matrix $\mathbf{B}_\mathrm{r}(\omega) \in \mathbb{R}^{Q \times Q}$ as
\begin{equation}\label{A_pq B_pq matrix form}
    \mathbf{A}_\mathrm{r}(\omega) + \frac{\mathrm{i}\mathbf{B}_\mathrm{r}(\omega)}{\omega}=2\pi \rho (\mathbf{C}_0 + \mathbf{C} \ \mathbf{X}) =2\pi \rho (\mathbf{C}_0 + \mathbf{C} \ \mathbf{A}^{-1} \ \mathbf{B})
\end{equation}
where the element in the $p$th row and $q$th column of $\mathbf{C}_0$ is $c_{pq}$, the $p$th row of $\mathbf{C}$ is $\vec{c}_p$, the $q$th column of $\mathbf{X}$ is $\vec{x}_q$, and the $q$th column of $\mathbf{B}$ is $\vec{b}_q$. Finally, the added mass and radiation damping matrices are 
\begin{equation}\label{eq: added mass matrix}
    \mathbf{A}_\mathrm{r}(\omega) = 2\pi \rho \ \mathrm{Re} \{ \mathbf{C}_0 + \mathbf{C} \ \mathbf{A}^{-1} \ \mathbf{B} \}
\end{equation}
and
\begin{equation}\label{eq: added mass matrix}
    \mathbf{B}_\mathrm{r}(\omega) = 2\pi \rho \omega \ \mathrm{Im} \{ \mathbf{C}_0 + \mathbf{C} \ \mathbf{A}^{-1} \ \mathbf{B} \} ,
\end{equation}
respectively. 

To find the heave excitation force $X_q$ on the $q$th body due to an incident wave, a form of the Haskind relation can be used~\cite{newman2018marine}. This was done in~\cite{chau2012inertia} and~\cite{triple_cylinder_WEC} for geometries with two and three internal regions, respectively. The results are the same for this configuration, as the derivation for the force on the $q$th body only involves the solution in the external region when the $q$th body is heaving. The heave excitation force on the $q$th body is
\begin{equation}\label{eq:excitation force}
    X_q = \frac{-4 \text{i} \rho g h \sqrt{N_0} \ _{}^{q}C_{10}^e}{\cosh(\lambda_0^eh)\text{H}_0^1(\lambda_0^e a_M)}
\end{equation}
where $g$ is the acceleration due to gravity, $\lambda_0^e$ is the wavenumber, $\textrm{H}_0^1$ is the zeroth-order Hankel function of the first kind, and $_{}^{q}C_{10}^e$ is the eigenfunction in the external region for $n=0$ when only the $q$th body is heaving. By applying Eq.~\ref{eq:excitation force} for $q=1,2,...,Q$, we can find the heave excitation force coefficient vector $\vec{X} \in \mathbb{C}^Q$, which is a column vector with elements $X_q$ in the $q$th entry.

% The magnitude and phase are
% \begin{equation}\label{eq:gamma-K}
%     |X_q|  = \sqrt{\frac{ 4 \rho g V_g  B_{mq}} {\lambda_0^e}} \quad \text{and} \quad % excitation
%    \angle X_q = -\frac{\pi}{2} + \angle\frac{ C_{10}^e}{\textrm{H}_0^{1}(\lambda_0^e a_M)}
% \end{equation}
% where $g$ is the acceleration due to gravity, $\lambda_0^e$ is the wavenumber, $V_g$ is the finite depth group velocity, $\textrm{H}_0^1$ is the zeroth-order Hankel function of the first kind, and $C_{10}^e$ is the eigenfunction in the external region for $n=0$. Note that while the excitation magnitude $|\gamma|$ depends on the radiation damping $B_h$, which in turn depends on all the inner region eigen-coefficients, the excitation phase $\angle\gamma$ depends only on the first exterior eigen coefficient, $C_{10}^e$. Eq.~\ref{eq:gamma-K} holds for any region heaving, but the solution to $\mathbf{A} \vec{x} = \vec{b}$ will be different, changing the values of $B_h$ and $C_{10}^e$ accordingly. 

The hydrostatic stiffness matrix $\mathbf{K}$ and mass matrix $\mathbf{M}$ are diagonal matrices that can be found from geometry. The element in the $q$th row and $q$th column of $\mathbf{K}$ can be found by summing over the waterplane areas $W_m$ contributed by each region
\begin{equation}\label{eq:hydrostatic stiffness}
    [\mathbf{K}]_{qq}= \rho g \sum_{m \in \mathcal{M}_q} W_m
\end{equation}
where $W_m= \pi a_m^2$ for $m=1$ and $W_m= \pi(a_m^2-a_{m-1}^2)$ otherwise.
If we assume the configuration in Fig.~\ref{fig:Diagram} is in static equilibrium such that the gravitational force balances the buoyancy force, the element in the $q$th row and $q$th column of $\mathbf{M}$, is
\begin{equation}\label{eq:mass matrix}
    [\mathbf{M}]_{qq}= \rho \sum_{m \in \mathcal{M}_q} W_md_m,
\end{equation}
which is a consequence of Archimedes' principle. This is the mass of the $q$th body. Once all matrices are found, one can construct the equation of motion of the system for regular waves
\begin{equation}\label{eq: EOM}
    (\mathbf{M} + \mathbf{A}_\mathrm{r}(\omega)) \ddot{\vec{\xi}} + \mathbf{B}_\mathrm{r}(\omega) \dot{\vec{\xi}} + \mathbf{K} \vec{\xi} =\mathrm{Re} \{ A \vec{X} e^{-\mathrm{i} \omega t} \}
\end{equation}
where $\vec{\xi}(t) = [\xi_1(t), \xi_2(t), ..., \xi_Q(t)]^T$ is a vector containing the heave displacements of the $Q$ bodies of the system, $\omega$ is the wave frequency, and $A$ is the wave amplitude.

\subsection{Numeric}
\begin{itemize}
    \item Overflow and scaling. Vertical overflow in MDOcean. Radial overflow. \textcolor{orange}{Bimali} and \textcolor{teal}{Becca}. 
    \item Limits on $m_k$ solve in MDOcean \textcolor{teal}{Becca}.
\end{itemize}

\subsection{Low, High, and Infinite Frequency Approximations}
\subsubsection{Low Frequency}


\subsubsection{High Frequency}
The $\sinh$ component of $N_0$ (and therefore $N_0$) increases exponentially with high $\lambda_0^e h$. $N_0$ appears in the first exterior region vertical eigenfunction $\frac{\cosh(\lambda_0^e(z + h))}{\sqrt{N_0}}$ and its derivative $\frac{\lambda_0^e \sinh(\lambda_0^e(z + h))}{\sqrt{N_0}}$. Anywhere else $N_0$ appears is a specific case of one of these expressions, so it is sufficient to find their limiting forms. The accuracy of these forms depends solely on the product $\lambda_0^e h$, not $\lambda_0^e $ or $h$ individually, or $z$.

\begin{equation}
	\lim_{\lambda_0^e h \to \infty} (\frac{\cosh(\lambda_0^e(z + h))}{\sqrt{N_0}}) = 
\sqrt{2 \lambda_0^e h} * (e^{\lambda_0^e z} + e^ {-\lambda_0^e(z + 2h)})
\end{equation}
\begin{equation}
	\lim_{\lambda_0^e h \to \infty} (\frac{\sinh(\lambda_0^e(z + h))}{\sqrt{N_0}}) = 
\sqrt{2 \lambda_0^e h} * (e^{\lambda_0^e z} - e^ {-\lambda_0^e(z + 2h)})
\end{equation}

Empirically, the approximated expressions are less than a fraction of $10^{-10}$ off from their true values for $\lambda_0^e h > 14$. This was encoded as the threshold for using the approximations.

\subsubsection{Infinite Frequency}
As $\lambda_0^e$ increases, the contribution/coefficient of the first exterior region eigenfunction decreases and is transferred to later eigenfunctions. At $\lambda_0^e = \infty$, it is no longer a valid eigenvalue, its eigenfunction is not valid either (has contribution 0) and the solution is representable with the rest of the eigenfunctions. 

The rest of the exterior region eigenvalues must be finite and satisfy $\lambda_n^e \tan (\lambda_n^e h) = - \infty$, meaning $\lambda_n^e h = (n - \frac{1}{2})\pi$ and
\begin{equation} \lim_{\lambda_n^e \to \infty} = \frac{(n - \frac{1}{2})\pi}{h}.
\end{equation}
In general, that is the lower bound for $\lambda_n^e$. For finite frequency, $\lambda_n^e h \in [ (n - \frac{1}{2})\pi, n\pi]$, bounds that can be passed into a root-finding solver for $\lambda_n^e$.

Lastly, damping and excitation phase approach zero as $\lambda_n^e$ approaches infinity. [TODO FINISH UP]


\begin{itemize}
    \item Low and infinite frequency approximations. Written by \textcolor{orange}{Bimali}.
    \item Figure showing that MEEM and Capytaine approach these values for low and infinite frequencies. Figure made by \textcolor{orange}{Bimali}.
\end{itemize}


\section{Convergence of Hydrodynamic Coefficients}\label{sec:convergence}
\subsection{Influential Parameters}
\begin{itemize}
    \item For cylindrical geometries without slants, discuss what dimensionless parameters have the most influence on the number of terms needed for the hydrodynamic coefficients to reach 1\% convergence. The dimensionless parameters are $m_0h$, $d_i/h$, $(r_{i}-r_{i-1})/h$, where $i$ indicates the region number of a multi-region geometry. Written by \textcolor{orange}{Bimali}.
    \item Figure showing the correlation between the number of terms needed for the hydrodynamic coefficients to reach 1\% convergence and the influential parameters. Figure made by \textcolor{orange}{Bimali}.
\end{itemize}
\subsubsection{Theory \& Procedure}
To predict the number of terms ($N^{i_m}$) needed in a region $m$ for a desired degree of accuracy, we assume this can be characterized by a set of local and cumulative dimensionless parameters. Local parameters are extracted from the geometry of the region itself, such as its fluid height, radial width, or the ratio of its fluid height to those of its neighbors. Cumulative parameters are overall metrics combining the geometries of other regions, such as total radial distance to the center/exterior region. They should not depend on a specific other region or the number of other regions. Sets of dimensionless parameters are not unique, but good sets will have more direct dependencies on each parameter individually.
[TODO transition/section reorder?]

To find the convergence of a configuration’s hydro coefficients with respect to $N^{i_m}$, compute them with all $N^{i_k}$ at a high number $N_{max}$ for their “true values”. Then, fix all $N^{i_k}, k\neq m$ to $N_{max}$, compute hydro coefficients with $N^{i_m}$ over a range of $[1, 2, …, N_{big}], N_{big}<N_{max}$, and find the errors $\epsilon = \frac{\text{val}-\text{true}}{\text{true}}$ of the hydro coefficients.

To predict errors, we model them as an envelope of $\epsilon \approx (\frac{x}{\beta})^{-\alpha}$, $\alpha, \beta > 0$. Convergence is faster with larger $\alpha$ and smaller $\beta$. To find the dependence of $\alpha$ and $\beta$ on geometry, we considered configurations with three body regions (and one exterior region), and found a good set of dimensionless parameters. After a dominant dimensionless parameter was found, it was kept constant in the search for the next most important parameter. After a set was decided, we kept as many dimensionless parameters constant as possible when measuring a single parameter's effect. Effects were quantified by fitting an $\alpha$ and $\beta$ value to each configuration (geometry + target region) and plotting them against the dimensionless parameter.

\subsubsection{Results}
The most influential dimensionless parameter for the convergence of a given region is its $\frac{\text{fluid height}}{\text{radial width}} = \frac{h - d_{i_m}}{a_{i_m} - a_{i_{m-1}}}$. For our set of eigenfunctions, the radial eigenfunctions inherit their eigenvalues $\lambda_n^{i_m} = \frac{n\pi}{h-d_m}$ from the corresponding vertical eigenfunction. The characteristic length of an eigenfunction is the inverse of its eigenvalue. Therefore, tall regions require more eigenfunctions to resolve the detail on surfaces of the same radial width.

The ratio of region's neighbor's fluid heights to that of its own also influences convergence. We observed that convergence is significantly faster for region $m$ when $\frac{h-d_{m-1}}{h-d_m} > 1$ than when it's less than $1$. There is some correlation before and after this transition point, but the convergence speed changes quickly (almost step-like) near $1$. The behavior is the same for $\frac{h-d_{m+1}}{h-d_m}$.

This is theorized to result from the boundary condition represented by the ratio. If $\frac{h-d_{m-1}}{h-d_m} < 1$, then region $m$'s entire inner boundary condition is the continuity conditions between regions $m$ and $m-1$, but if $\frac{h-d_{m-1}}{h-d_m} > 1$, the BC includes a section of $\frac{\partial \phi}{\partial r} = 0$ near $z = d_m$, where the hydro coefficients are being integrated. [more explanation needed]

[Shielding effects]
Especially in the case the outer neighboring region the, the impact of fluid height ratio might be a special case of what we've termed a "shielding effect". 

[wavenumber * height]

\subsection{Limiting Factors}
\begin{itemize}
    \item For cylindrical geometries without slants, identify which dimensionless parameters determine the required number of terms needed for the hydrodynamic coefficients to reach 1\% convergence. Written by \textcolor{orange}{Bimali}.
    \item Table showing the required number of terms needed for the hydrodynamic coefficients to reach 1\% convergence as a function of key parameters. Figure made by \textcolor{orange}{Bimali}.
\end{itemize}



\section{Slanted Geometries}\label{sec:slant}
\subsection{Corrections for Slanted Geometries}
\begin{itemize}
    \item Discuss ways that the equations in the previous subsections can be modified to better approximate slanted regions. Written by \textcolor{orange}{Bimali}.
    \item Diagram showing how slanted geometries can be approximated by a finite number of annual cylinders. Figure made by \textcolor{blue}{Collin}.
\end{itemize}


\begin{figure}
    \centering
    \includegraphics[width=0.95\linewidth]{figs/slant diagram.pdf}
    \caption{Side view and close up of a slanted body. The close-up shown on the right-hand side shows how a slanted region can be approximated by a finite number of cylindrical rings. The unit vector normal to the horizontal part of all cylindrical rings is $\hat{n}$, while the unit vector normal to the true slanted body in the $m$th region is $\hat{n}_{\beta_m}$. \textcolor{blue}{Collin, replace this with updated figure.}}
    \label{fig:Diagram}
\end{figure} 



\subsection{Discretizations of Slanted Geometries}
\begin{itemize}
    \item Given a specified profile $z=f(r)$ that is revolved around the axis of symmetry and a specified number of regions to approximate the slanted geometry, determine a way to discretize the slanted geometry. That is, find $r_i$ and $d_i$ that best approximate the geometry. If this has already been done in literature, we can just use their method. Written by \textcolor{orange}{Bimali}.
    % \item Using the number of terms recommended from the previous section for each region, determine the number of regions needed for the hydrodynamic coefficients to reach 1\% convergence.
    \item Case studies of two cones with different slopes. Number of regions vs. accuracy. Written by \textcolor{orange}{Bimali}.
    % \begin{itemize}
        % \item For 2-3 slanted geometries (sphere, RM3, etc.), show plots of the number of regions required for the hydrodynamic coefficients to reach 1\% convergence as a function of dimensionless parameters that describe the geometry. Figure made by \textcolor{orange}{Bimali}.
    %     \item Discuss figure. Written by \textcolor{orange}{Bimali}.
    % \end{itemize}
\end{itemize}

% \subsection{Non-Cylindrical Geometries}
% \subsubsection{Influential Parameters}
% \begin{itemize}
%     \item Discuss what parameters have the most influence on the number of terms needed for the hydrodynamic coefficients to reach 1\% convergence. The possibilities for the limiting factors are $m_0h$, $d/h$, $r/h$, where $d$ is the draft of a indicates the region number of a multi-region geometry. Written by \textcolor{orange}{Bimali}.
%     \item Figure showing the correlation between the number of terms needed for the hydrodynamic coefficients to reach 1\% convergence and the limiting factors.
% \end{itemize}

% \subsubsection{Limiting Factors}
% \begin{itemize}
%     \item Identify which parameters determine the required number of terms needed for the hydrodynamic coefficients to reach 1\% convergence.
%     \item Figure showing the required number of terms needed for the hydrodynamic coefficients to reach 1\% as a function of key parameters.
% \end{itemize}


\section{Validation}
\subsection{Non-Slanted Geometries}
\begin{itemize}
    \item Figures showing added mass, radiation damping, real part of excitation force, and imaginary part of excitation force (vertical axis) as a function of frequency (horizontal axis). One line for MEEM that was converged using the previous section's recommendations, and one line for BEM that is also converged. Do this for a multi-region geometry. Figure made by \textcolor{orange}{Bimali}.
    \item Discuss these figures, and explain any errors. Written by \textcolor{orange}{Bimali}.
\end{itemize}

\subsection{Slanted Geometries}
\begin{itemize}
    \item Figures showing added mass, radiation damping, real part of excitation force, and imaginary part of excitation force (vertical axis) as a function of frequency (horizontal axis). One line for MEEM that was converged using the previous section's recommendations, and one line for BEM that is also converged. Do this for 2-3 slanted geometries (sphere, RM3, etc.). Figure made by \textcolor{orange}{Bimali}.
    \item Discuss these figures, and explain any errors. Written by \textcolor{orange}{Bimali}.
\end{itemize}

\section{Computation Time and Accuracy}\label{sec:compute-time}
\begin{itemize}
    \item Figure showing what takes the longest to compute. \textcolor{orange}{Bimali}.
    \item Discussion about reusing things to save computation time. \textcolor{orange}{Bimali}.
    \item For a specific \textbf{non-slanted geometry}, make figures showing accuracy of added mass, radiation damping, real part of excitation force, and imaginary part of excitation force (vertical axis) as a function of runtime (horizontal axis). Two lines per figure: one for MEEM and one for BEM. Each point on the figure corresponds to a different level of accuracy. Written and figures made by \textcolor{orange}{Bimali}.
    \begin{itemize}
        \item For BEM, vary the number of panels to change the level of accuracy. 
        \item For MEEM, vary the \textbf{number of terms per region} to vary the level of accuracy. Also, highlight a point on the graph where the number of terms used is the recommended amount for that geometry.
        \item Discuss the computational advantages of MEEM.
    \end{itemize}
    % \item For a specific \textbf{slanted geometry}, make figures showing added mass, radiation damping, real part of excitation force, and imaginary part of excitation force (vertical axis) as a function of runtime (horizontal axis). Two lines per figure: one for MEEM and one for BEM. Each point on the figure corresponds to a different level of accuracy. Written and figures made by \textcolor{orange}{Bimali}.
    % \begin{itemize}
    %     \item For BEM, vary the number of panels to change the level of accuracy.
    %     \item For MEEM, vary the \textbf{number of regions} to vary the level of accuracy. Use the number of terms per region that was recommended in the convergence section. Also, highlight a point on the graph where the number of regions used is the recommended amount for that geometry.
    %     \item Discuss the computational advantages of MEEM.
    % \end{itemize} 
    \item State that, for a non-slanted geometry, MEEM can be within XX\% of the true hydrodynamic coefficient, with a XX times faster computation time than BEM. Written by \textcolor{orange}{Bimali}.
\end{itemize}

 The $\lambda_n^e$ only depend on $\omega$ and $h$. Since a common use case is evaluating many geometries at a fixed location (subject to the same depth and over the same frequency range), the $\lambda_n^e$ computation can be cached. Similarly, changing which region(s) are heaving only changes entries in the b-vector, meaning both the A-matrix and a significant part of the matrix solve can be reused.

Time complexities for MEEM and BEM are ultimately both cubic (since both involve a matrix solve), MEEM with the total number of unknown coefficients and BEM with the number of panels in its mesh. Comparing these values (or their cubes) is a good proxy for time when they are high and repeated trials/timing under like conditions is difficult. [TODO: fix these commentaries if BEM is square in the operating range]

[FIGURES]
[TODO: RM3 convergence figure, only float heaving].
[Caption: MEEM and BEM matrix size comparison, using the RM3 configuration parameters, an optimal unknown coefficient distribution for MEEM, and even constant panel density throughout the mesh for BEM. MEEM reaches 1\% convergence XX\% faster/slower than BEM for added mass and XX\% faster for radiation damping.]
[TODO: MEEM is worse convergence figure, e.g. skinny spar]
[Caption: The $\frac{\text{fluid height}}{\text{radial width}}$ in the spar region is very high and requires more terms to resolve its effects, while BEM requires even less panels than the typical RM3 configuration due to the decreased surface area. Here, BEM reaches 1\% convergence XX\% faster than MEEM for added mass, and XX\% faster for radiation damping.]
[TODO: MEEM is far better convergence figure (short height? longer spar? unless CPT is glitchy on this, check plots)]


\section{Conclusion}\label{sec:conclusion}
\begin{itemize}
    \item Summarize main findings and contributions of this work. Written by \textcolor{teal}{Becca}.
    \item Discuss areas of future work. Written by \textcolor{teal}{Becca}.
\end{itemize}



\section{Code Availability}
Open source (check JFM)

% \begin{thebibliography}{9}
% \expandafter\ifx\csname natexlab\endcsname\relax\def\natexlab#1{#1}\fi

% \bibitem[{{Arntzenius and Dorr}(2012)}]{Arntzenius2012}
% {Arntzenius, F. and Dorr, C.} (2012)  Calculus as Geometry in {\em Space, Time,
%   and Stuff}. Oxford University Press.
% \newblock \doi{https://doi.org/10.1093/9780199696604}.

% \bibitem[{{Eddon}(2013)}]{eddon_fundamental_????}
% {Eddon, M.} (2013) Fundamental properties of fundamental properties In {\em
%   Oxford Studies in Metaphysics, Volume 8},  Bennett, K. and Zimmerman, D.
%   (eds). vol.~8. Oxford University Press. pp. 78--104.
% \newblock \doi{https://doi.org/10.1093/9780199682904}.

% \bibitem[{{Field}(1980{\natexlab{a}})}]{Field1980}
% {Field, H.} (1980{\natexlab{a}}) {\em Science Without Numbers}. Princeton
%   University Press.
% \newblock \doi{https://doi.org/10.1093/molbev/msy092}.

% \bibitem[{{Field}(1980{\natexlab{b}})}]{Field1980y}
% {Field, H.} (1980{\natexlab{b}}) {\em Second Science Without Numbers}.
%   SPrinceton University Press.
% \newblock \doi{https://doi.org/10.1007/s13194-011-0027-5}.

% \bibitem[{{Field}(1980{\natexlab{c}})}]{Field1980z}
% {Field, H.} (1980{\natexlab{c}}) {\em Third Science Without Numbers}.
%   TPrinceton University Press.
% \newblock \doi{https://doi.org/10.1086/508108}.

% \bibitem[{{Field}(1984)}]{field_can_1984}
% {Field, H.} (1984) Can we dispense with space-time? \textit{{FLM:} Proceedings
%   of the Biennial Meeting of the Fluid Mechanics}. {\it
%   1984}, 33--90.
% \newblock \doi{https://doi.org/10.1086/192496}.

% \bibitem[{{H\"{o}lder}(1901)}]{Hoelder1901}
% {H\"{o}lder, O.} (1901) Die axiome der quantit\"{a}t und die lehre vom mass
%   (part 1). \textit{Journal of Mathematical Psychology}. {\it 40}(23),
%   235--252.

% \bibitem[{{Mundy}(1987)}]{mundy_metaphysics_1987}
% {Mundy, B.} (1987) The metaphysics of quantity. \textit{Philosophical Studies}.
%   {\it 51}(1), 29--54.
% \newblock \doi{https://doi.org/10.1007/bf00353961}.

% \bibitem[{{Perry}(2015)}]{PerryForthcoming-PERPEQ}
% {Perry, Z.~R.} (2015) {P}roperly {E}xtensive {Q}uantities. \textit{{P}hilosophy
%   of {S}cience}. {\it 82}, 833--844.
% \newblock \doi{https://doi.org/10.1086/683323}.

% \end{thebibliography}

% \cite{mundy_metaphysics_1987,Field1980,Field1980y,Field1980z,field_can_1984,PerryForthcoming-PERPEQ,Hoelder1901,eddon_fundamental_????,Arntzenius2012}
%
\bibliographystyle{jfm}
\bibliography{jfm}

%%%%%%%%%%%%%%%%%%%%%%%%%%%%%%%%%%%%%%%%%%%%%%%%%%%%%%%%%%%%%%%%%%%%%%%%%%%%%%%%%%%%%%%%%%%%%%%%%%%%%%%%%%%%%%%%%%%%%%%
%%%%%%%%%%%%%%%%%%%%%%%%%%%%%%%%%%%%%%%%%%%%%%%%%%%%%%%%%%%%%%%%%%%%%%%%%%%%%%%%%%%%%%%%%%%%%%%%%%%%%%%%%%%%%%%%%%%%%%%


\section{Citations and references}

All papers included in the References section must be cited in the article, and vice versa. Citations should be included as, for example ``It has been shown \citep{Rogallo81} that...'' (using the {\verb}\citep}} command, part of the natbib package) ``recent work by \citet{Dennis85}...'' (using {\verb}\citet}}).
The natbib package can be used to generate citation variations, as shown below.\\
\verb#\citet[pp. 2-4]{Hwang70}#:\\
\citet[pp. 2-4]{Hwang70} \\
\verb#\citep[p. 6]{Worster92}#:\\
\citep[p. 6]{Worster92}\\
\verb#\citep[see][]{Koch83, Lee71, Linton92}#:\\
\citep[see][]{Koch83, Lee71, Linton92}\\
\verb#\citep[see][p. 18]{Martin80}#:\\
\citep[see][p. 18]{Martin80}\\
\verb#\citep{Brownell04,Brownell07,Ursell50,Wijngaarden68,Miller91}#:\\
\citep{Brownell04,Brownell07,Ursell50,Wijngaarden68,Miller91}\\
\citep{Briukhanovetal1967}\\
\cite{Bouguet01}\\
\citep{JosephSaut1990}\\

The References section can either be built from individual \verb#\bibitem# commands, or can be built using BibTex. The BibTex files used to generate the references in this document can be found in the JFM {\LaTeX} template files folder provided on the website \href{https://www.cambridge.org/core/journals/journal-of-fluid-mechanics/information/author-instructions/preparing-your-materials}{here}.

Where there are up to ten authors, all authors' names should be given in the reference list. Where there are more than ten authors, only the first name should appear, followed by {\it {et al.}}


\section{Miscellaneous section heads}

Philosophy of Science asks authors to include Acknowledgments, Declarations (of competing interests), and Funding Information in your typeset manuscript.  Please add three sections, reflecting each of these categories, using ``bmhead'' coding as shown below.

\begin{verbatim}
\begin{bmhead}[Xxxxxxx.]
For the custom heading such as acknowledgment, funding disclosure,
conflict disclosure and any other like-wise sections must be
mentioned in the optional braces as shown in this example.
\end{bmhead}
\end{verbatim}
The output of the above coding is shown below:

\begin{bmhead}[Xxxxxxx.]
For the custom heading such as acknowledgment, funding disclosure,
conflict disclosure and any other like-wise sections must be
mentioned in the optional braces as shown in this example.
\end{bmhead}

% \backsection[Supplementary data]{\label{SupMat}Supplementary material and movies are available at \\https://doi.org/10.1017/jfm.2019...}
%
% \backsection[Acknowledgements]{Acknowledgements may be included at the end of the paper, before the References section or any appendices. Several anonymous individuals are thanked for contributions to these instructions.}
%
% \backsection[Funding]{Please provide details of the sources of financial support for all authors, including grant numbers. Where no specific funding has been provided for research, please provide the following statement: "This research received no specific grant from any funding agency, commercial or not-for-profit sectors." }
%
% \backsection[Declaration of interests]{A Competing Interests statement is now mandatory in the manuscript PDF. Please note that if there are no conflicts of interest, the declaration in your PDF should read as follows: {\bf Declaration of Interests}. The authors report no conflict of interest.}
%
% \backsection[Data availability statement]{The data that support the findings of this study are openly available in [repository name] at http://doi.org/[doi], reference number [reference number]. See JFM's \href{https://www.cambridge.org/core/journals/journal-of-fluid-mechanics/information/journal-policies/research-transparency}{research transparency policy} for more information}
%
% \backsection[Author ORCIDs]{Authors may include the ORCID identifers as follows.  F. Smith, https://orcid.org/0000-0001-2345-6789; B. Jones, https://orcid.org/0000-0009-8765-4321}
%
% \backsection[Author contributions]{Authors may include details of the contributions made by each author to the manuscript'}


%\appendix
\begin{appen}

\section{}\label{appA}
In order not to disrupt the narrative flow, purely technical material may be included in the appendices. This material should corroborate or add to the main result and be essential for the understanding of the paper. It should be a small proportion of the paper and must not be longer than the paper itself.

\end{appen}\clearpage


%\bibliographystyle{jfm}
%\bibliography{jfm}

%Use of the above commands will create a bibliography using the .bib file. Shown below is a bibliography built from individual items.


% \begin{thebibliography}{}
% \expandafter\ifx\csname natexlab\endcsname\relax
% \def\natexlab#1{#1}\fi
% \expandafter\ifx\csname selectlanguage\endcsname\relax
% \def\selectlanguage#1{\relax}\fi

% \bibitem[Batchelor (1971)]{Batchelor59}
% {\sc Batchelor, G.K.} 1971 {Small-scale variation of convected quantities like temperature in turbulent fluid part1, general discussion and the case of small conductivity}, {\it J. Fluid Mech.}, {\bf 5}, pp. 3-113-133.

% \bibitem [Bouguet (2008)]{Bouguet01}
% {\sc Bouguet, J.-Y} 2008 Camera Calibration Toolbox for Matlab {\url{http://www.vision.caltech.edu/bouguetj/calib_doc/}}.

%  \bibitem[Briukhanovetal et al (1967)] {Briukhanovetal1967}
% {\sc Briukhanov, A. V.,   Grigorian, S. S., Miagkov,  S. M., Plam, M. Y.,   I. E. Shurova, I. E.,   Eglit, M. E. and Yakimov, Y. L.} 1967
% {On some new approaches to the dynamics of snow avalanches},
% {\it Physics of Snow and Ice,  Proceedings of the International Conference on Low Temperature Science}
% {Vol 1} pp. 1221--1241 {Institute of Low Temperature Science, Hokkaido University, Sapporo, Hokkaido, Japan}.

% \bibitem[Brownell (2004)]{Brownell04}
%  {\sc Brownell,  C.J.  and Su,  L.K.} 2004  {Planar measurements of differential diffusion in turbulent jets}, {\it AIAA Paper},  pp. 2004-2335.

% \bibitem[Brownell and Su (2007)] {Brownell07}
%   {\sc Brownell, C.J. and  Su, L.K.} 2007 {Scale relations and spatial spectra in a differentially diffusing jet}, {\it AIAA Paper}, pp 2007-1314.

% \bibitem [Dennis (1985)] {Dennis85}
%  {\sc  Dennis, S.C.R.} 1985 {Compact explicit finite difference approximations to the Navier--Stokes equation},  { In \it Ninth Intl Conf. on Numerical Methods in Fluid Dynamics},  {ed Soubbaramayer and J.P. Boujot},  {Vol 218}, {\it Lecture Notes in Physics}, pp. 23-51. Springer.

% \bibitem [Edwards et al. (2017)]{EdwardsVirouletKokelaarGray2017}
% {\sc Edwards, A. N., Viroulet, S., Kokelaar, B. P. and Gray, J. M. N. T.} 2017 Formation of levees, troughs and elevated channels by avalanches on erodible slopes {\it J. Fluid Mech.}, {\bf 823}, pp. 278-315.

% \bibitem[Hwang et al (1970)] {Hwang70}
%  {\sc Hwang,  L.-S.  and  Tuck, E.O.} 1970 On the oscillations of harbours of arbitrary shape {\it J.~Fluid Mech.}, {\bf42}, pp 447-464.

% \bibitem[Josep and Saut (1990)] {JosephSaut1990}
%  {\sc Joseph, Daniel D. and Saut, Jean Claude} 1990 Short-wave instabilities and ill-posed initial-value problems {\it Theoretical and Computational Fluid Dynamics}, {\bf 1},  pp.191--227,  {\url{http://dx.doi.org/10.1007/BF00418002}}.

% \bibitem[Worster (1992)] {Worster92}
% { \sc  Worster, M.G.} 1992 The dynamics of mushy layers {\it Interactive dynamics of convection and solidification},
% {(ed. S.H. Davis and H.E. Huppert and W. Muller and M.G. Worster)}, pp. 113--138 {Kluwer}.

% \bibitem[Koch(1983)] {Koch83}
% {\sc Koch, W.} 1983 Resonant acoustic frequencies of flat plate cascades {\it J.~Sound Vib.}, {\bf 88}, pp. 233-242.

% \bibitem[Lee(1971)] {Lee71}
% {\sc Lee,  J.-J.}  1971 Wave-induced oscillations in harbours of arbitrary geometry {\it J.~Fluid Mech.}, {\bf 45}, pp. 375-394.

% \bibitem[Linton and  Evans (1992)] {Linton92}
%  {\sc  Linton, C.M. and  Evans, D.V.} 1992 The radiation and scattering of surface waves by a vertical circular cylinder in a channel {\it Phil.\ Trans.\ R. Soc.\ Lond.}, {\bf 338}, pp. 325-357.

% \bibitem [Martin(1980] {Martin80}
%  {\sc  Martin, P.A.} 1980 On the null-field equations for the exterior problems of acoustics {\it Q.~J. Mech.\ Appl.\ Maths},{\bf 33}, pp. 385--396.

% \bibitem [Rogallo(1981)] {Rogallo81}
%  {\sc Rogallo,  R.S.} 1981 Numerical experiments in homogeneous turbulence  { {\it Tech. Rep.} 81835}  {NASA Tech.\ Mem}.

% \bibitem[Ursell(1950)] {Ursell50}
% {\sc  Ursell, F.} 1950 Surface waves on deep water in the presence of a submerged cylinder i {\it Proc.\ Camb.\ Phil.\ Soc.}, {\bf 46}, pp.141--152.

% \bibitem[Wijngaarden (1968)]{Wijngaarden68}
% {\sc van Wijngaarden, L.} 1968 On the oscillations Near and at resonance in open pipes {\it J.~Engng Maths},{\bf 2}, pp. 225--240.

% \bibitem[Miller (1991)]{Miller91}
% { \sc  Miller, P.L.} 1991 Mixing in high Schmidt number turbulent jets {school {PhD thesis}} {California Institute of Technology}.

% \end{thebibliography}

%% End of file `jfm.bib'.


\end{document}
